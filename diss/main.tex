\documentclass[12pt,a4paper,openany,openright]{report}
\usepackage[utf8]{inputenc}
\usepackage[T1]{fontenc}
\usepackage[margin=20mm]{geometry}
\usepackage[pdfborder={0 0 0},backref=page]{hyperref}
\usepackage{dissertation}

\include{math_commands}

\begin{document}

\newcommand{\mcandidate}{123456}
\newcommand{\mfullname}{Skye Purchase}
\newcommand{\mcollege}{Peterhouse}
\newcommand{\mtitle}{Extracting Concepts from Simplified Graph Neural Networks}
\newcommand{\mexamination}{Computer Science Tripos -- Part II}
\newcommand{\mdate}{\today}
\newcommand{\moriginator}{\mcandidate \& Pietro Lio}
\newcommand{\msupervisor}{Charlotte Magister and Pietro Babiero}
\newcommand{\mwordcount}{...}
\newcommand{\mlinecount}{...}
\newcommand{\mconsent}{I am content for my dissertation to be made available to the students and staff of the University.}
\newcommand{\msignature}{Skye Purchase}


\bibliographystyle{plain}

%%%%%%%%%%%%%%%%%%%%%%%%%%%%%%%%%%
% Title

\thispagestyle{empty}

\rightline{\LARGE \textbf{\mfullname}}

\vspace*{60mm}
\begin{center}
    \Huge
    \textbf{\mtitle} \\[5mm]
    \mexamination \\[5mm]
    \mcollege \\[5mm]
    \mdate
\end{center}

%%%%%%%%%%%%%%%%%%%%%%%%%%%%%%%%%
% Proforma

\pagestyle{plain}

\newpage
\newpage
\section*{Declaration of originality}

I, \mfullname{} of \mcollege, being a candidate for Part II of the Computer Science Tripos, hereby declare that this dissertation and the work described in it are my own work, unaided except as may be specified below, and that the dissertation does not contain material that has already been used to any substantial extent for a comparable purpose. \mconsent

\bigskip
\leftline{Signed \msignature}
\bigskip
\leftline{Date \today}

\chapter*{Proforma}

{\large
\begin{tabular}{p{0.3\linewidth}p{0.6\linewidth}}
    Candidate Number:   & \bf \mcandidate                   \\
    Project Title:      & \bf \mtitle                       \\
    Examination:        & \bf \mexamination, 2023           \\
    Word Count:         & \bf \mwordcount\footnotemark[1]   \\
    Code Line Count:    & \bf \mlinecount\footnotemark[2]   \\
    Project Originator: & \moriginator                      \\
    Supervisor:         & \msupervisor                      \\ 
\end{tabular}
}

\footnotetext[1]{This word count was computed by \texttt{texcount} using the options \texttt{\%group table 0 1} and \texttt{\%group tabular 1 1} to count tables.}
\footnotetext[2]{This word count was computed by \texttt{cloc}. Only generating bash scripts where considered as the generated files are all structurally identical.}
\stepcounter{footnote}

\section*{Original Aims of the Project}

\error{Include why the reader should care}
This project analyses the effect that simplifying graph neural networks (GNNs), through linearisation, has on model performance and why this is the case.
%the graph concepts that are extracted from a trained model.
The project focuses on the linear GNN, Simplified Graph Convolution \cite{wu2019simplifying}, as it has been shown to match the performance of traditional, non-linear GNNs on a sample of benchmark datasets.
%which removes the non-linearity between layers from previous graph neural network approaches reducing the problem of graph representational learning to a precomputation on the graph structure and a single linear regressor.
Using the notion of graph concepts proposed in \textit{Magister et. al} \cite{magister2021gcexplainer} the project aims to compare SGC to Graph Convolution Network to determine the effect of linearising GNNs.
The overall aim of the project is to better understand how linear architectures can achieve graph structure awareness.
These insights then guide extensions to improve linear GNN architectures and reduce computational cost.
%Further extensions then build on SGC to see which techniques can improve accuracy and concept metrics.

\section*{Work Completed}

The project was a success gaining insight into the graph structure awareness of both linear and non-linear models.
The insight gained lead to two new approaches to graph learning
\begin{enumerate}
    \item Using both linear and non-linear GNNs in one model to gain the benefit of both approaches.
    \item A new linear GNN which aggregates linear graph filters to improve graph structure awareness.
\end{enumerate}
The project further demonstrated that the SGC architecture proposed by \textit{Wu et al.} does not have fine-grained graph structure awareness and demonstrates poor performance on a range of new datasets.
%Furthermore, it demonstrated that in the case of highly synthetic data SGC is unable to match the performance of GCN by a significant margin.
%My extensions demonstrate that in the case of real-world graph datasets, where graph structure is important, SGC continues to underperform compared to GCN and does not produce comparable concepts.
%To combat this a novel extension to SGC using jumping knowledge networks\cite{xu2018representation} is presented that demonstrates graph structure awareness in a linear model.
%However, overall SGC does not appear to be a suitable candidate for graph representational learning on its own.
%I additionally set out to create a new parameterised dataset to further analyse the shortcomings of SGC when dealing with graph structure.
%\error{This is still not complete but I am undertaking this task.}

\section*{Special Difficulties}

None

%%%%%%%%%%%%%%%%%%%%%%%%%%%%%%%%%
% Chapters

\pagestyle{headings}

\tableofcontents

\section{Introduction}

%\subsection{Motivation}
%\label{sec:motivation}

% Importance/rise in use of GNNs

% The rise in use of ML systems

%% Wide spread of ML systems in general prevalent in every day use

\emph{Neural networks} (NN) are able to infer complex relationships within data, \emph{graph neural networks} (GNN) extend this functionality to connected systems where predetermined relationships exist between data points.
In these systems, such as social networks~\cite{pmlr-v70-gilmer17a} and molecules~\cite{DBLP:journals/corr/abs-1806-01973}, the data can be represented as nodes connected by edges in a graph.
However, the deep learning approach, of multiple layers separated by non-linear activations, has been criticised as unnecessarily complex for GNNs.~\cite{wu2019simplifying}
This is because, unlike standard NNs, the power of GNNs is hypothesised to arise from utilising the graph topology.
Thus, the ability for deep NNs to approximate arbitrary functions using non-linearity may not be required.
Instead \emph{simplified GNNs} have been proposed~\cite{chanpuriya2022simplified,chien2020adaptive,wu2019simplifying} which are either entirely linear or quasi-linear, using one or two non-linear layers.

%\begin{figure}
%    \centering
%    \captionsetup{width=0.9\textwidth}
%    \includegraphics[width=0.8\textwidth]{figures/linear-vs-non-linear}
%    \caption{An overview of how traditional, non-linear GNNs work compared to linear GNNs. The distinction between GNN layers and graph filters is made as linearisation has focused on graph filters ignoring the more complex GNN layer approaches.}
%    \label{fig:linear-vs-non-linear}
%\end{figure}
%
%Figure \ref{fig:linear-vs-non-linear} demonstrates the difference in architecture between traditional, non-linear GNNs and linear GNNs.
These linear models have been shown to match the performance of non-linear GNNs on a limited number of datasets.~\cite{wu2019simplifying, chanpuriya2022simplified, chien2020adaptive}
However, these datasets represent a small portion of possible graph datasets and little work has been done to understand how linear GNNs work.

\begin{figure}[h]
    \centering
    \captionsetup{width=0.9\textwidth}
    \includegraphics[width=0.95\textwidth]{figures/concept}
    \caption{An example graph concept from GCN~\cite{kipf2016semi} demonstrating that when classifying the top node the model identifies the house structure and attaching arm. The consistency of the structure suggests that this is important when classifying the top node.}
    \label{fig:concept}
\end{figure}

%\fig{concept}{An example graph concept from GCN~\cite{kipf2016semi} demonstrating that when classifying the top node the model identifies the house structure and attaching arm. The consistency of the structure suggests that this is important when classifying the top node.}

This lack of understanding extends generally to NNs where explaining how a model works is largely ignored in favour of higher accuracy.
%This approach is flawed as it is possible to create explainable architectures that do not limit accuracy~\cite{zarlenga2022concept}.
This paper combats this by analysing NNs through the subspaces in their input which influence their classification the most, these are referred to as \emph{concepts}.
This approach avoids altering the architecture and thus maintains the original accuracy achieved by the trained model.
An example of a concept is presented in Figure \ref{fig:concept} which can be interpreted as ``a house structure with an attaching arm''.
This explains how the GNN classifies the highlighted node, by identifying the house structure and attaching arm, in a human-interpretable way.

Concepts can highlight the limitations of linear GNN models, by demonstrating which graph structures and node features cannot be distinguished.
By comparing the differences in concepts between linear and non-linear models further insight into how linearity influences graph structure awareness can be gained. 
This motivates the paper which aims to provide a deeper insight into graph representational learning and provide novel techniques to graph structure inference.

%In recent years there has been rapid development and adoption of machine learning systems such as \emph{neural networks} (NN).
%However, methods to explain how these ever larger models work is lagging behind leading to either mistrust in, or worse, blind trust in these systems.
%Though standard NNs are in the mainstream spotlight there is increasing requirement for NNs to infer knowledge about connected systems such as social networks and molecules.
%The mainstream advancements result in increasingly larger \emph{neural network} (NN) models focusing on text and image based input.
%This development makes sense from the perspective of human interaction however these are limited datastructures especially in the ever increasing interaction of digital information.
%Data present in social networks, computational biology, and systems such as smart cities have multiple predefined connections between each data point.
%In these systems the connected structure is important to classification of the individual data-points leading to the concept of \emph{graph data} and the idea of the \emph{graph neural network} (GNN).
%But criticisms of the deep learning (DL) approach to GNNs has resulted in \emph{Linear GNNs} (SGNNs) that remove non-linearity and focus on the additional information provided by the graph connections.
%\note{somewhere else: How do these linear GNNs compare to non-linear GNNs in how they infer labels for graph data?}
%\note{somewhere else: Are there potential insights into how graph structure can be better utilised?}
%\note{somewhere else: This dissertation answers these questions by evaluating an SGNN within an explanability framework and provides new methods of utilising graph data in SGNNs.}

%% Brief description of evolution of GNNs, looking at motivation and use cases

%\paragraph{Graph data}
%Rather than a dataset being only a collection of feature vectors representing observed data points within the problem setting an additional adjacency matrix is also present.
%The adjacency matrix represents the connections between data points that is inherently present in the observed data.
%An example being friendship connections in a social network or bonds between atoms in a molecule.
%This additional structure provides useful information for tasks where the interaction between data points has importance to the data points themselves.
%More complex graph data may also include attributes associated with the connections which can be simple scalars or multidimensional vectors.
%For these reasons graph data is a highly flexible and versatile datastructure promoting complex inference.

%\paragraph{Graph neural networks (GNNs)} 
%are designed to handle graph data where the connections between data points is an important aspect as the data itself.
%The standard form of a GNN 
%\note{make sure this is valid here!}
%consists of multiple layers connected together by a non-linearity step.
%Each layer performs inference on a node's feature vector in the same way as a NN would perform inference.
%The graph structure is utilised by broadcasting this new node representation along the connected edges to neighbouring nodes.
%Each node then aggregates the representations of its neighbouring nodes creating a compact representation of the neighbourhood according to itself.
%The updated representation and aggregation are then combined to produce a final representation before the next layer.
%By using a process known as \emph{message passing} each node's feature vector is updated based on the graph neighbourhood inferring graph structure as well.
%This process, known as \emph{message passing}, allows a GNN to utilise the graph structure and infer more complex relationships between the data points than an ordinary NN.

%% Rapid integration of these systems 

% The lack of clarity in ML systems
% -> Explainability
% -> Concepts

%\paragraph{Linearising NNs}
%Modern specialist computer hardware, such as graphics processing units (GPU), are incredibly efficient at carrying out large linear operations such as matrix multiplication.
%However, the vast majority of NNs contain non-linear operators between generally linear layers.
%The idea of linearising NNs is to remove \emph{some} of the non-linearity in the architecture multiple layers to be combined into a single linear operation.
%Removal the early non-linearity (or all the non-linearity) results in a pre-computation step that can be carried out on an entire dataset before training or inference.
%%Thus in cases where inputs are sampled from a collection of data points this prevents calculating the same operation on the same data point across samples.
%But, it is important that this does not effect the performance of the model.

%\paragraph{Linear GNNs}
%The process of message passing though complex conceptually can easily be decomposed into a linear operation on the graph features and graph adjacency.
%Using the idea of linearising NNs the non-linearity between individual message passing layers can be removed.
%This allows a GNN to pre-compute multiple message passing steps on the graph dataset before inference.
%Inference then results in a simple classification or linear regression task where only a single layer is required.
%This new \emph{linear GNN} (SGNN) remove their non-linear complexities whilst demonstrating comparable performance to standard GNNs.

%\paragraph{Explainability}
%The advancement of new NNs has focused on improving the performance in metrics such as accuracy and training cost which has resulted in impressive models.
%However, these models remain as blackboxes to the users of these systems but equally to the designers.
%Once a model is trained on a specific dataset there is very limited understanding of how the model is analysing the input to produce a result.
%This can and does create a lot of mistrust in NNs as there is a large element of trusting that the output it produces on unseen data will match our expectation.
%The idea of explainability is to provide different methods of visualising how a model works to a human as a form of verification or to provide insight.
%Many different approaches exist including before, during and after training, to varying degrees of explanation.

%% Importance of understanding ML inference
%% -> Link to potential use cases 

%% Issues with interfering with training

%% Recent development in developing frameworks to analyse these aspects
%% The different goals of explainability 

%\paragraph{Concepts}
%A specific form of explainability that is applied after the training of a model \note{Check that this is always the case!} is that of concepts.
%The idea is to find different subspaces of the input space (where each input to tthe NN is some element of the input space) that correspondent to a specific output.
%This way patterns can be found between input and output to verify that the model is behaving as expected.
%In this dissertation the focus is on graph concepts which are subgraphs created from the input graph(s) for the model.
%These subgraphs represent the graph structures that the model is using to carry out inference on specific inputs.
%
%\paragraph{Concepts in simplified GNNs}
%Though SGNNs show competitive performance when looking at classification accuracy they are a black box as with the majority of NNs.
%As SGNNs delay the influence of the models weights until the final classification step this provides an opportunity to see how graph concepts are effected.
%If these models are truly comparable to GNNs then the concepts should demonstrate how the message passing steps operate.
%These results could therefore influence the design of future networks as the DL approach may not be required for graph data. \note{This needs better wording, I don't know how much I agree with it myself.}

%% The idea of a concept
%% The fact in our case this is done after training
%% -> This does not effect training performance

% More efficient systems should not sacrifice ease of understanding

%This project applies the graph explainability tools proposed by \citet{magister2021gcexplainer} to the linear model SGC~\cite{wu2019simplifying}.
This paper provides a new analysis of linear GNN models and their limitations providing new insight into how graph structure is inferred.
These results demonstrate that purely linear models, such as SGC, cannot infer fine-grained graph structure and are instead optimised for tabular data.
We hypothesise that this limitation is due to limited access to intermediate node representations during classification.

In this work we make the following contributions:
\begin{itemize}
    \item 
        We demonstrate that the current approaches to linearising GNNs are limited by their inability to infer graph structure.
    \item 
        We introduce SGCN, a quasi-linear model that exploits the efficient SGC pre-computation but retains some GCN layers to maintain high accuracy.
    \item
        We introduce JSGC, a quasi-linear model that provides the classification layer with intermediate node representations to demonstrate that this is a major bottleneck in linear models.
\end{itemize}


\chapter{Preparation}

\section{Background}

\section{Graph Representational Learning}

% Formal definition of graph based data

%% Formal definition of a graph

A \emph{graph} can be formally represented as the tuple $\gG(\sV, \sE, \mX)$ where $\sV$ is the set of nodes in the graph, $\sE$ is the set of edges, and $\mX$ the data associated with the graph also known as the input features.
Two nodes of the graph, $v_i, v_j \in \sV$, are \emph{connected} if and only if $(v_i, v_j) \in \sE$, the edge is directed with $v_i$ as the source and $v_j$ as the destination.
%It is possible to associate data with the edges of the graph where $(v_i, v_j, \textbf{e}_{ij}) \in \sE$ represents a connection between $v_i$ and $v_j$ and associated vector $\textbf{e}_{ij}$.
Each node, $v_i \in \sV$, has associated data represented by the \emph{feature vector} $\vx_i = \emX_{i,\star}$ the $i^{th}$ row of $\mX$.

The edges can be represented as an \emph{adjacency matrix}, $\mA$, where 

\begin{equation*}
\emA_{ij} = \begin{cases}1 &\text{if $(v_i, v_j) \in \sE$} \\ 0 &\text{otherwise}\end{cases}.
\end{equation*}

This allows the edge information to be passed to a NN as another feature vector along with $\mX$.
Furthermore, each node, $v_i \in \sV$, has \emph{neighbours} which is set of connected nodes, $\sN_i = \{v_j | (v_i, v_j) \in \sE\}$.
An element of $\sN$ is a therefore a \emph{neighbour} of the node $v_i$.
%If there is data associated with the edges, such as edge weights, this value is stored in the adjacency matrix.
%An unweighted graph can be described as a weighted graph with edge weights of value $1$.
%Adjacency matrices can be very sparse, meaning they have very few non-zero values, and therefore the matrix is commonly stored in a \emph{coordinate matrix} (COO) of size $2 \times N$ for $N$ edges in the graph thus only storing edge information.

% Formal definition of representational learning

The goal of graph representational learning is to find, for each $v_i \in \sV$, a vector $\vh_i$ based on the the input features $\mX$ and adjacency matrix $\mA$.
$\vh_i$ is known as the \emph{node representation} and $\mH$ is the total graph representation where $\mH_{i, \star} = \vh_i$.

% Comparison of graph, edge and node based learning

%% Expand the above definition to edges and graphs
%% discussion of edge weights in training when talking about edge learning
%% discussion of pooling functions with graph classification

These representations may then be used as input to a classifier to solve one of 3 different classification tasks: node classification, graph classification, edge classification.
For the remainder of this dissertation only node and graph classification will be considered.
Node classification classifies each node solely on the final node representation $\vh_i$.
Comparitively graph classification aggregates all of the node representations and classifies the entire graph accordingly.

\subsection{Graph Neural Networks}

%% Formal definition of a function on this graph
%%      Think about Petars talk here

To produce updated node representations consider a function, $F$, which acts on a graph taking the input features, $\mX$, and adjacency matrix, $\mA$, and producing a new feature matrix $F(\mX, \mA)$.
If this function where to be given instead a different permutation of the graph it is important that the resulting output is permuted the same way.
That is given some permutation, $\mP$, the following must hold

\begin{equation}
    F(\mP\mX, \mP\mA) \equiv \mP F(\mX, \mA).
\end{equation}

This function can be described by a function, $f$, which acts on a single node, $v_i$, using the feature vector, $\vx_i$, and neighbours, $\sN_i$.
Rather than $f$ being applied to the input features consider some node representation $\vh_i^l$ after $l$ iterations, where $\vh_i^0 = \vx_i$.
Then $\vh_i^{l+1} = f^l(\vh_i^l, \sN_i)$, where $f^l$ is the $l^{th}$ application of the function $f$.

%% The idea of invariants, equivariants and approaches

% Formal definition of a graph neural network

%% Expand on the invariants and equivariants above (maybe only start that here)
%% Formalise the notion of how a GNN would behave
%% Discuss the differing approaches

% Potentially the motivation behind this formalisation

%% This is the invariance and equivariance
%% Might be possible to motivate this from the development perspective

% Identify the root of deep learning in its formulation

%% Highlight the concept of non-linear layers
%% potentially a brief discussion on the importance of this concept

\subsection{Graph Convolutional Network}

% Formal implementation of Graph Convolutional Network

%% Expand on the convolutional approach to discuss this formulation

\subsubsection{As a GNN}

The \emph{graph convolutional network} (GCN) follows the formulation of GNNs outlined above the simplest fashion possible.
The principle is to add the edge weighted sum of a node's neighbour representations to the current node representation.
This maintains the graph function equivariance constraint as sum is equivariant, however, nodes with a high degree are overly represented and so each term is normalised by the degree of the connecting nodes.
This results in the common representation of GCN as

\begin{equation}
    f(\vh_i^l) = \frac1{d_i + 1}\vh_i^l + \sum_j^N\frac{\emA_{ij}}{\sqrt{(d_i + 1)(d_j + 1)}}\vh_j^{l},
\end{equation}

where $d_i$ is the degree of node $v_i$, $d_i = \sum_j^N \emA_{i,j}$.

To better demonstrate that the same operation is being applied to each neighbouring representation (including the nodes current representation) the equation may be rewritten as

\begin{equation}
    f(\vh_i^l) = (d_i + 1)^{-\frac12}1(d_i + 1)^{-\frac12}\vh_i^l + \sum_j^N(d_i + 1)^{-\frac12}\emA_{ij}(d_i + 1)^{-\frac12}\vh_j^{l}.
\end{equation}

Let $\mD$ be the degree matrix, $\emD_{ii} = \sum_j^N \emA_{ij}$, and $\mH^l$ the node representations of the graph after layer $l$. The above equation can now be compactly described as 

\begin{equation}
    F(\mH^l) = (\mD + 1)^{-\frac12}(\mA + \mI)(\mD + 1)^{-\frac12}\mH^l.
\end{equation}

Let $\widetilde{\mA} = \mA + \mI$, then $\widetilde{\mD} = \sum_j^N \widetilde{\mA} = \sum_j^N[\mA] + 1 = \mD + 1$.
Ignoring the application and looking only at the operator which will be called $\mS$, the other representation of a GCN layer is
\begin{equation}
    \mS = \widetilde{\mD}^{-\frac12}\widetilde{\mA}\widetilde{\mD}^{-\frac12}.
\end{equation}

\subsubsection{As a convolution}

The above formulation is often derived from the principle of a convolution applied to the graph.
From this view point $\widetilde{\mA}$ represents a renormalisation of the convolution by adding self-loops where every node in the graph is connected to itself.
The motivation is to prevent exploding/vanishing weights that occur from the approximate graph convolution

\begin{equation}
    \mS = \mI + \mD^{-\frac12}\mA\mD^{-\frac12}
\end{equation}

which has eigenvalues in the range $[0,2]$ rather than remaining at or close to $1$.

\error{How do convolutions and spectral filters arrive at this stage?}

%= \widetilde{\textbf{D}}^{-\frac12}\widetilde{\mA}\widetilde{\textbf{D}}^{-\frac12}\mX\textbf{\Theta}$

% Link to the paper

%% Lay out the exact formulation from the paper
%% Explain this formulation
%% include the motivations/reasonings from the paper

% Discussion of importance in the field?

%% Description of the fact that this is the first graph approach
%% Maybe a link to DeepSet

\subsection{Simplified Graph Convolution}

% Motivation behind SGC

%% Reiterate the points made in the introduction
%% Now specifically highlight these areas in the GCN formulisation

% Demonstration of how the GCN leads to the SGC

%% Demonstrate that the removal of layers logically leads to SGC
%% Explanation of how the weight matrix would behave
%% Potential hint or discussion about the effect of weight decay later

\section{Explainability}

% Overview of the concept of explainability

%% This needs some proper reading in the topic see below

% This probabily requires reading some papers "yay"

\subsection{Concepts}

% Explanation of what a concept is generally

%% Again this requires proper reading in the subject see above

% Relating the explanation to graph based Datasets

%% Demonstrate the idea of concepts translates to subgraphs
%% Focus on the node classification task for this explanation
%% Potentially a motivating example

% Discussion of complications when looking at graph classification

%% Identify the issue when scaling to graph classification
%% Discuss solutions and the chosen solution for this project

\subsection{Graph Concept Explainer}

% The motivation for the paper

%% The human in the loop aspect of this project
%% Link in ideas from GNNExplainer

% Discussion of the metrics

%% Formulisation of the two metrics that are presented
%% Discussion what each metric measures with link to results
%% Discussion of short comings of clusterings
%% Discussion of further benefits of latent space

% Detailing why this method was chosen

%% The improvements seen in the paper in regards to visual comparison
%% The clear metrics for comparison

\section{Datasets}

% Continue the concepts in GRL subchapter
% There may be more details about implementation
% May need chapters here or somewhere else about inductive v. transductive!!

\subsection{Synthetic}

% From GRL and the overview discuss a bottom up approach
% Discuss how certain properties can be instilled
% Relation to graph theory or importance thereof

\subsection{Real-World}

% Linking to motivation and GRL
% Demonstrating that these techniques are important for real world use
% Discussions of the short comings of this approach

\section{Tools Used}

% I believe this is stuff like vim
% Pytorch, pytorch lightning and pytorch geometric
% scipy and numpy
% python unittest and typing
% python linter through vim

%MAYBE THE REQUIREMENTS ANALYSIS POST-HUMOUSLY?

% definitely think this is useful
% to be figured out later

\section{Software Methodology}

% Generally waterfall approach to design
% But iterative when looking at the results and direction forward
% sprint work style between meetings

\section{Testing}

% unittest for specific sections of code and infrastructure
% comparison to paper baselines when dealing with model implementation

\section{Licensing}

% Definitely need to research this aspect

\section{Starting Point}

% Discussion of summer project using the same rough build environment

\chapter{Implementation}

\section{Datasets}
\label{sec:datasets-imp}
\Sref{sec:datasets-theory} presents a number of datasets that cover a range of different GNN training styles and motifs.
The transductive datasets discussed in \Sref{sec:RWD} are already packaged and available through \texttt{PyTorch Geometric}\cite{Fey/Lenssen/2019} as well as the Planetoid\cite{planetoid}\cite{citation} datasets.

The synthetic datasets discussed in \Sref{sec:synth} are proposed by \textit{Ying et al.}\cite{ying2019gnnexplainer}.
Though downloadable versions of these datasets are available from their github and \texttt{PyTorch Geometric} provide limited versions of a small subset I decide to implement them myself.
This allows for more rigorous testing of both the datasets and the SGC precomputation later described in \Sref{sec:testing-imp}.

Though the datasets are very simple they need to be compliant with \textit{PyTorch Geometric} to be used correctly during training.
\textit{PyTorch Geometric} provides an \texttt{InMemoryDataset} abstract class for datasets that are small enough to be generated and stored within RAM.

\error{Finish this!}

\section{Models}

\section{Testing}
\label{sec:testing-imp}

\section{Machine Learning Pipeline}

\subsection{Reproducibility}

\subsection{Experimentation}

\section{Concept Extraction and Evaluation}

\subsection{Extraction}

\subsection{Evaluation}

\subsection{Visualisation}

\section{Extensions}
\note{Include a short preamble.}
The evaluation of my extensions is presented in \Sref{sec:extension-eval}.

\subsection{SGC graph classification}
\error{Make sure this makes sense to flow.}
\paragraph{Motivation}
The datasets present in \textit{Magister et al.}\cite{magister2021gcexplainer} include 2 real world datasets that focus on graph classification.
As discussed in \Sref{sec:datasets} these graph classification datasets are also inductive rather than transductive which provides another test of the capabilities of SGC.
Furthermore, though \Sref{sec:comp-acc} suggests SGC performs poorly, this could be because of the synthetic nature of the datasets.

\paragraph{Prior work and implementation}
\textit{Wu et al.}\cite{wu2019simplifying} discuss graph classification for SGC suggesting that it can substitute GCN in a deep graph convolutional neural network as proposed by \note{name needed}\note{citation needed}.
However, \textit{Magister et al.} utilise pooling on the graph node representations after GCN to classify graphs. As this latter method is easier to implement and allows for a fairer comparison between SGC and GCN this approach is chosen.
The resulting model is identical to the standard SGC model with addition of a pooling layer before the classifier.

The only problem posed is to concept extraction as this is done on a node level as suggested by \textit{Magister et al.}.
This is overcome by broadcasting the graph label to each of the nodes.
This allows the graphs to be combined into a disconnected forest of graphs and clustering can be carried out on this forest.
Calculation of concept scores and the visualisation of concept therefore remains the same.

\subsection{SGC and GCN mixed model}
\error{Make sure this makes sense to flow.}
\paragraph{Motivation}
SGC is directly derived from GCN, the derivation provided in \Sref{sec:SGC}, using the same graph filter.
The benefit of SGC is to reduce the complexity of GCN and the cost of training by pre-computing the graph operation.
Though for the presented datasets in \Sref{sec:datasets} the improvement on cost and reduction in parameters is not significant larger datasets may create a larger benefit.
However, due to the low accuracy of SGC some aspects of GCN must required for high accuracy, therefore a mixture of both should yield a high accuracy model with pre-computation.
This combined model, \textit{SGC+GCN} (SGCN), maintains the same hyperparameters as the SGC and GCN models with the additional hyperparameter of whether a layer is SGC or GCN.

\paragraph{Implementation}
For simplicity and to utilise the power of pre-computation SGCN starts with SGC layers and then transitions to GCN layers.
For a fairer comparison the total number of layers must remain the same as the corresponding SGC and GCN layers.
This reduces the problem to finding where SGC and GCN agree the most in regards to their concepts.

This is achieved using mutual information between the two models by using the probability of a node appearing in a specific cluster from either model.
This gives a measure for the dependence of the models clusters and therefore how similar the models are.
However, this does not take into account random chance of two nodes appearing in the same cluster.
Therefore the mutual information is adjusted for this chance resulting in a number in the range $[0, 1]$ where $1$ is identical.
This is known as \emph{adjusted mutual information}(AMI).

To better visualise these results the dimensionality of the activations from the models are reduced using \emph{t-distributed stochastic neighbor embedding} (t-SNE)\note{citation needed} into 2 dimensions.
This clusters similar representations together and keeps different representations apart which acts as a visual proxy to viewing all the concepts of a model.
These can then be compared to see why a specific layer has the most mutual information and how the join effects the resulting model.

\subsection{Jumping knowledge SGC}
\error{Make sure this makes sense to flow.}
\paragraph{Motivation}
As discussed in \Sref{sec:concept-analysis} SGC does not have influence on the node representations during message passing.
SGC, of degree $k$, therefore has to infer graph structure from the aggregated node representations of all neighbouring nodes within $k$ hops.
Therefore the reason for the low accuracy and poor graph structure awareness may not be due to the linearity.
GCN is able to manipulate node representations between graph convolutions and can therefore further distinguish graph structure by potentially amplifying or damping differences between nodes.

For these reasons I propose \emph{jump-SGC}(JSGC) which provides the classifier with node representations from each degree of the pre-computation.
A fully-connected layer is added before the classifier to reduce the concatenated node representations into a single node representation.
This allows for JSGC to effectively manipulate the node representations though these manipulations do not have an impact on the application of the graph filter.
This idea mimics the \emph{jumping knowledge networks} (JCNs) proposed by \textit{Xu et al.}\cite{xu2018representation} hence the name ``jump''.

\paragraph{Prior work and Implementation}
\textit{Xu et al.}\cite{xu2018representation} identify the drawbacks of node aggregation in accurately representing a nodes neighbourhood.
It specifically identifies the effect on graph structure awareness this has making the method ideal for SGC.
The motivation for JCNs is the node aggregation methods used resulted in neighbourhood influence similar to a random walk rather than a uniform influence.
As a solution they propose aggregating the the node representations after successive neighbourhood aggregation layers together.
Three main methods of aggregation are proposed but given the small size of the datasets the proposed concatenation method is best suited.

By concatenating successive neighbourhood aggregations and then reducing the dimensionality to a single node representation uniform influence can be achieved.
This is because detail present in the closer neighbourhoods can be combined with the wider awareness of the more receptive neighbourhoods.
Rather than missing larger structure awareness or missing detail JCNs allow for an analysis of both.

For JSGC this leads to two changes to the model and pre-computation.
During pre-computation successive applications of the normalised filter are concatenated together.
A fully-connected layer is then added to the standard SGC to reduce this concatenated representation space to the standard representation space.
During this stage JSGC is able to infer more complex graph structure than SGC.
To combine this with the classifier a single non-linear rectified linear unit layer is introduced.
This non-linearity remains constant regardless of how the model scales and therefore the added potential benefits of the single non-linear layer is deemed negligible.

\section{Pytorch Geometric}

\section{Repository}

\chapter{Evaluation}

\section{Success criteria}

\paragraph{Success Criterion}
The original project proposal (\Aref{ch:proposal}) stated the following three criteria for success:
\begin{enumerate}
    \item 
        Implement SGC and extract the concepts used for each of the synthetic datasets.
        \label{crit1}
    \item 
        Implement GCN and extract the concepts to use as a baseline.
        \label{crit2}
    \item 
        Compare the concepts between SGC and GCN using the metrics of concept completeness and concept purity.
        \label{crit3}
\end{enumerate}

\emph{I completely meet all three success criteria.}
In addition to the above project success criteria, and to aid the analysis of SGC compared to GCN, 
I compare the two models to each other in mean \emph{test accuracy}.

\paragraph{Meeting criterion 1}
\note{reference the implementation.}
\Sref{sec:reproduction} verifies the correctness of the SGC implementation.
\Sref{sec:comp-concept} demonstrates concept extraction on the synthetic datasets.

\paragraph{Meeting criterion 2}
\note{reference the implementation.}
\Sref{sec:reproduction} verifies the correctness of the GCN implementation and demonstrates concept extraction.

\paragraph{Meeting criterion 3}
\note{reference the implementation.}
\Sref{sec:comp-concept} demonstrate the comparison of SGC and GCN using the metrics of concept completeness and purity.
Additionally, \Sref{sec:comp-acc} demonstrates further comparison between the models using the metric of model accuracy.

\section{Methodology}

\subsection{Hyperparameters}
\label{sec:hyperparameters}

\paragraph{Reproduction}
\textit{Magister et al.}\cite{magister2021gcexplainer} use a GCN model to evaluate their proposed graph concept explainer.
The paper trains and evaluates the model on the same 5 synthetic node classification datasets described in \Sref{sec:synth} and therefore the same hyperparameters are used for the GCN baseline.

\textit{Wu et al.}\cite{wu2019simplifying} states that the weight decay parameter for the Planetoid datasets was found using \texttt{hyperopt} over 60 iterations.
This process was repeated for results reproduction however it was found that the hyperparameters for the learning rate were different from those stated.

The hyperparameters for these models are available in tables \ref{tab:GCN-params} and \ref{tab:SGC-reproduction-params}.
Additionally the concept extraction metrics are presented in \ref{tab:GCN-concept-params}.

\begin{table}
    \centering
    \begin{tabular}{c|ccc}
        \textbf{Dataset} &
        \textbf{Layers} &
        \textbf{Learning rate} &
        \textbf{Epochs} \\
        \midrule
        BA Shapes       & 3 $\times$ 20 & 0.001 & 3000 \\
        BA Grid         & 3 $\times$ 20 & 0.001 & 3000 \\
        BA Community    & 6 $\times$ 50 & 0.001 & 6000 \\
        Tree Cycles     & 3 $\times$ 50 & 0.001 & 7000 \\
        Tree Grid       & 7 $\times$ 20 & 0.001 & 10000 \\
        \midrule \\
        REDDIT BINARY   & 4 $\times$ 40 & 0.005 & 3000 \\
        Mutagenicity    & 4 $\times$ 30 & 0.005 & 10000 \\
    \end{tabular}
    \caption{GCN hyperparameters from table 15 in \textit{Magister et al.}\cite{magister2021gcexplainer}. All models use \texttt{ReLU}\note{citation needed?} activation functions between layers with a single linear layer classifier with \texttt{softmax}\note{citation needed?}.}
    \label{tab:GCN-params}
\end{table}


\begin{table}
    \centering
    \begin{tabular}{c|cccc}
        \textbf{Dataset} &
        \textbf{Degree} &
        \textbf{Learning rate} &
        \textbf{Weight decay} &
        \textbf{Epochs} \\
        \midrule
        Cora        & 2 & 0.002 & 0.04 & 100 \\
        Citeseer    & 2 & 0.004 & 0.08 & 100 \\
        PubMed      & 2 & 0.0005 & 0.25 & 100 \\
    \end{tabular}
    \caption{SGC hyperparameters using \texttt{hyperopt}\note{citation needed} as proposed by \textit{Wu et al.}\cite{wu2019simplifying}. All models use \texttt{softmax}\note{citation needed?}.}
    \label{tab:SGC-reproduction-params}
\end{table}


\begin{table}
    \centering
    \begin{tabular}{c|cc}
        \textbf{Dataset} &
        \textbf{Clusters (k)} &
        \textbf{Receptive field (n)} \\
        \midrule
        BA Shapes       & 10 & 2 \\
        BA Grid         & 10 & 3 \\
        BA Community    & 30 & 2 \\
        Tree Cycles     & 10 & 3 \\
        Tree Grid       & 10 & 3 \\
        \midrule
        REDDIT BINARY   & 20 & 1 \\
        Mutagenicity    & 30 & 3 \\
    \end{tabular}
    \caption{Concept extraction parameters used in \textit{Magister et al.}\cite{magister2021gcexplainer}.}
    \label{tab:GCN-concept-params}
\end{table}



\paragraph{New models}
This project proposes multiple new models designed for the different datasets.
The core project defines 5 models based on the SGC architecture as required by criterion \ref{crit1}.
As the underlying graph operator is the same for both GCN and SGC, as described in \Sref{sec:GCN} and \Sref{sec:SGC} the degree of the SGC model can be assumed to be the same as the GCN model for the same dataset.
This the same approach that \textit{Wu et al.}\cite{wu2019simplifying} take to creating there SGC models.

The learning rate for SGC models is likely to be different from the GCN models due to the reformulation.
Furthermore, \textit{Wu et al.}\cite{wu2019simplifying} use weight decay to keep weight values close to $0$ as would be assumed from the multiplication of weight matrices in equation \ref{eq:theta}.
Rather than use stochastic approaches to finding these hyperparameters a sweep of anticipated values is carried out.
This allows for a visualisation of the hyperparameter allowing for further exploration if necessary.
The learning rate is sampled from $\{0.01, 0.001, 0.0001\}$ and the weight decay from $\{1.0, 0.1, 0.01\}$.

The results of these searches is presented in figures \note{include the references the hyperparameter surfaces}.
As can be seen for the majority of hyperparameter searches the specific hyperparameters have minimal to no impact on the resulting model accuracy and therefore $0.01$ is chosen for learning rate and $0.1$ for weight decay.
The hyperparameters chosen are presented in table \ref{tab:SGC-params}

\begin{table}
    \centering
    \begin{tabular}{c|cccc}
        \textbf{Dataset} &
        \textbf{Degree} &
        \textbf{Learning rate} &
        \textbf{Weight decay} &
        \textbf{Epochs} \\
        \midrule
        BA Shapes       & 3 & 0.01 & 0.1 & 3000 \\
        BA Grid         & 3 & 0.01 & 0.1 & 3000 \\
        BA Community    & 6 & 0.001 & 0.1 & 6000 \\
        Tree Cycles     & 3 & 0.01 & 1.0 & 7000 \\
        Tree Grid       & 7 & 0.01 & 0.01 & 10000 \\
    \end{tabular}
    \caption{SGC hyperparameters based on the corresponding GCN hyperparameters in \ref{tab:GCN-params} and using the results of the hyperparameter search in \note{reference when added}. All models use \texttt{softmax}\note{citation needed?}.}
    \label{tab:SGC-params}
\end{table}



\paragraph{Datasets}
The batch sizes for all the datasets match those described in \textit{Ying et al.}\cite{ying2019gnnexplainer} and \textit{Kipf et al.}\cite{kipf2016semi}.
The number of epochs for each dataset matches those proposed in \textit{Magister et al.}\cite{magister2021gcexplainer} and \textit{Wu et al.}\cite{wu2019simplifying} or until convergence for SGC.

\subsection{Model Evaluation}
\label{sec:evaluation}

\paragraph{Concept evaluation}
Criterion \ref{crit3} requires evaluation of the models with respect to concept purity and completeness.
These metrics are discussed in \ref{sec:GCE} and implementation is described in \note{reference when complete}.
On top of the quantitative analysis of the different models concepts lend themselves to qualitative analysis which will mainly focus on visual similarities between the two models.
the quantitative analysis will also help to infer the differences in how the two models infer labels on the input data.

For qualitative analysis only BA Shapes and Mutagenicity will be covered in detail with brief analysis of the other datasets in Appendix \note{reference when complete}.
This extends to extensions where only BA Shapes and Mutagenicity are trained on as proofs of concepts.

It is important to note a number of drawbacks of the GCExplainer in comparing two different models quantitatively.
\begin{enumerate}
    \item 
        Concept purity is calculated only using subgraphs with less than 13 nodes.
        This means that pure quantitative analysis does not represent a full comparison of the two models.
    \item 
        The number of clusters and the receptive field of the concepts can be arbitrarily manipulated to find the highest score.
        To combat this the new SGC models are compared against the same concept extraction parameters presented in table \note{reference when completed}.
    \item 
        \label{nb:accuracy}
        \textit{Ying et al.}\cite{ying2019gnnexplainer} only suggest concept extraction for models that achieve an accuracy of atleast $95\%$ on synthetic datasets. 
\end{enumerate}

The full comparison of concepts requires a qualitative analysis of the extracted concepts.
The concepts produced by SGC can be analysed in isolation to infer how SGC reasons about graphs.
These can then be compared to the concepts produced by GCN to highlight the differences in reasoning and which is easier to understand.
When reproducing results this is done by comparing the analysis suggested by \textit{Magister et al.}\cite{magister2021gcexplainer} to the reproduced concepts.
Furthermore, a visual comparison of the concepts can be made, such as matching published concepts to reproduced concepts.

\paragraph{Accuracy evaluation}
Drawback \ref{nb:accuracy} motivates the additional evaluation metric of accuracy as \Sref{sec:comp-acc} demonstrates that SGC does not meet the desired accuracy.
To evaluate this each synthetic dataset is split into a train and test set using an 80:20 split.
Note that TUDataset\note{citation needed} and Planetoid\note{citation needed} use there own train/test splits.

The synthetic datasets are generated randomly along with the train/test split and thus each random seed produces a new variation of the synthetic dataset.
This means that using the same seeds across different models for the experiments results in the same train/test split.
This also means that although the hyperparameter search evaluates parameters on the test split of the dataset using a different seed to the experiments means that the experiment test splits are effectively unseen.

\subsection{Confidence intervals}
\label{sec:reporting}
The mean accuracy across 10 different initiliasations is reported using $\mu = \sum_i\frac{\text{accuracy}_i}{10}$ as an unbiased estimator of the mean.
The confidence interval of each of the runs uses the unbiased standard deviation estimator $\sigma = \sqrt{\sum_i(\text{accuracy}_i - \mu)/(10 - 1)}$.
For experiments where very high variance is present outliers are removed based on the median, $m$, and the interquartile range, $\text{ITR}$, of the accuracies.\footnote{In the cases where outliers are removed the estimators are adjusted accordingly.}
An outlier is defining as being outside the range $[m - 1.5 \times \text{ITR}, m + 1.5 \times \text{ITR}]$.


\subsection{Reproducibility}
\error{Return to at the end to make sure that all aspects are met.}

\subsection{System specifications}
The experiments are not resource-intensive due to the incredibly small datasets and so carrying out the hyperparameter search and multiple final runs can be completed on my personal machine.
My machine has an AMD Ryzen 7 5700U CPUs @ 1.8GHz with 16 cores wuth 15 Gigabytes of RAM.
The machine does have an AMD ATI Lucienne GPU but due to the fact that \texttt{PyTorch Geometric}\note{citation needed} did not support RoCM I was unable to utilise this.

To speed up the retrieval of experimental results for extensions I utilised a Google Colab Pro account with 1 hyperthreaded Intel Xeon Processor @ 2.3GHz with 1 core and 12 Gigabytes of Ram.
The account also has access to a Tesla K80 GPU with 12GB of RAM.

\note{Include figures demonstrating system use later.}

\section{Results Reproduction}
\label{sec:reproduction}

As discussed in \Sref{sec:testing} the testing strategy includes the reproduction of prior results from \textit{Magister et al.}\cite{magister2021gcexplainer} on GCN and \textit{Wu et al.}\cite{wu2019simplifying} on SGC.
I reproduce all of the experiments from \textit{Magister et al.} and the Planetoid\cite{kipf2016semi} from \textit{Wu et al.} as these are most relevant.

\paragraph{Success criterion}
\Sref{sec:GCN-reproduction} demonstrates an implementation of GCN trained on the synthetic datasets with concept extraction.
Tables \ref{tab:GCN-acc} and \ref{tab:GCN-concepts} and figure \ref{fig:GCN-BA-Shapes} demonstrate the results achieved.
This fulfills the \ref{crit2}$^{nd}$ success criterion.

\paragraph{Method}
In both cases across all datasets the hyperparameters presented in tables \ref{tab:GCN-params} and \ref{tab:SGC-reproduction-params} are used.
Each model, dataset experiment is run 10 times with different randomly pre-selected seeds and the mean and confidence interval are presented in accordance with \Sref{sec:reporting}.
The implementation of SGC is the one presented in \note{reference section when complete} and the implementation of GCN uses the layers provided by \texttt{PyTorch Geometric}\cite{Fey/Lenssen/2019}.

\subsection{SGC}
\begin{table}
    \centering
    \begin{tabular}{c|cc}
        \multicolumn{1}{c}{\textbf{Dataset}} &
        \multicolumn{1}{c}{\textbf{Wu et al.}} &
        \multicolumn{1}{c}{\textbf{My results}} \\
        \midrule
        Cora        & 81.0\% $\pm$ 0.0 & 80.0\% $\pm$ 0.1 \\
        Citeseer    & 71.9\% $\pm$ 0.1 & 71.4\% $\pm$ 0.1 \\
        Pubmed      & 78.9\% $\pm$ 0.0 & 71.3\% $\pm$ 2.1 \\
    \end{tabular}
    \caption{Reproduction of experiments from table x in \textit{Wu et al.}\cite{wu2019simplifying}. The mean accuracy of 10 experiment runs is taken using hyperparameters found by \texttt{hyperopt}\note{citation needed}.}
    \label{tab:GCN-acc}
\end{table}



Table \ref{tab:SGC-reproduction} presents the accuracy achieved by my SGC models compared to the accuracy presented in table 2 of \textit{Wu et al.}\cite{wu2019simplifying}.
As can be seen in both Cora and Citeseer the accuracies are closely correlated, though the accuracy presented by \textit{Wu et al.} are outside of my confidence interval.
Comparitively Pubmed presents a large discrepancy between the published results and the reproduced results with large variation in the reproduced results.
These descrepancies are likely due to the uncertainty in hyperparameters as the exact weight decay constant used is unknown.
Furthermore, the published learning rate of 0.2 yields worse results and so, as demonstrated in table \ref{tab:SGC-reproduction-params}, new learning rates are used.
Based on these considerations I consider my implementation of SGC to be correct.

\subsection{GCN}
\label{sec:GCN-reproduction}
\paragraph{Accuracy}
\begin{table}
    \centering
    \begin{tabular}{c|cc}
        \multicolumn{1}{c}{\textbf{Dataset}} &
        \multicolumn{1}{c}{\textbf{Magister et al.}} &
        \multicolumn{1}{c}{\textbf{My results}} \\
        \midrule
        BA Shapes       & 95.6\% & 98.0\% $\pm$ 2.2 \\
        BA Grid         & 99.0\% & 99.5\% $\pm$ 0.7 \\
        BA Community    & 95.7\% & \% $\pm$  \\
        Tree Cycles     & 96.0\% & 95.8\% $\pm$ 3.2 \\
        Tree Grid       & 95.1\% & 92.5\% $\pm$ 5.4 \\
        \midrule
        REDDIT BINARY   & 89.1\% & 89.1\% $\pm$ 2.2 \\
        Mutagenicity    & 93.0\% & 79.3\% $\pm$ 1.7 \\
    \end{tabular}
    \caption{Comparison of test accuracy published by \textit{Magister et al.}\cite{magister2021gcexplainer} and those achieved by myself.}
    \label{tab:GCN-acc}
\end{table}

%\begin{enumerate}
%    \item[S1] Implementation of SGC
%    \item[S2] Implementation of GCN
%    \item[S3] Implementation of concept evaluation
%\end{enumerate}


Table \ref{tab:GCN-acc} presents the accuracy achieved my GCN models compared to the accuracy published in table 16 of \textit{Magister et al.}\cite{magister2021gcexplainer}.
As can be seen in the majority of cases the reproduced accuracy matches or exceeds the accuracy presented by \textit{Magister et al.}.
Given the small size of the synthetic datasets it is expected that close to 100\% accuracy is achieved which is demonstrated in all but BA Community.
For the real-world datasets the expectation is above 85\% as suggested by \textit{Ying et al.}\cite{ying2019gnnexplainer}.

BA Community is a significant outlier in the results neither reaching the suggested 95\% or reaching a value close to 100\%.
If the model is allowed to train for more than the defined 6000 epochs an accuracy closer to 95\% is achieved.
A likely cause for this discrepancy is the uncertainty in the construction of the community dataset discussed in \note{reference when complete}.
Due to these considerations I consider my implementation of GCN to be correct.

\paragraph{Concept scores}
\begin{table}
    \centering
    \begin{tabular}{c|cccc}
        \multirow{2}{*}{\textbf{Dataset}} &
        \multicolumn{2}{c}{\textbf{Magister et al.}} &
        \multicolumn{2}{c}{\textbf{My results}} \\
        &
        \textbf{Completeness} & 
        \textbf{Purity} & 
        \textbf{Completeness} & 
        \textbf{Purity} \\
        \midrule
        BA Shapes       & 0.964 & 3.375 & 0.929 & 0.000 \\
        BA Grid         & 1.000 & 4.923 & 1.000 & 0.000 \\
        BA Community    & 0.678 & 0.000 & 0.623 & 5.600 \\
        Tree Cycles     & 0.949 & 1.167 & 0.970 & 4.391\\
        Tree Grid       & 0.965 & 3.100 & 0.951 & 1.417\\
        \midrule
        REDDIT BINARY   & 0.713 & 6.968 & 0.746 & 1.429 \\
        Mutagenicity    & 0.967 & 5.400 & 0.626\tablefootnote{See footnote \ref{nb:epochs1}} & 0.375 \\
    \end{tabular}
    \caption{Reproduction of experiments from tables 4 and 5 in \textit{Magister et al.}\cite{magister2021gcexplainer}. The activation space for each experiment selected from the best performing model is used with the mean purity score presented.}
    \label{tab:GCN-concepts}
\end{table}


Table \ref{tab:GCN-concepts} presents the concept scores for each of the top performing GCN models compared to those in tables 4 and 5 in \textit{Magister et al.}.
Only the mean purity is presented across all the models with the minimum and maximum purity available in table \note{reference when complete}. As with the accuracy reproduction the values for concept completeness are closely correlated for the majority of the models.

\fig{erroneous-labels}{A subset of the concepts discovered for Mutagenicity highlighting the erroneous labels. Each row represents a concept and the graphs are coloured according to standard chemical colouring \note{citation needed}.}

Figure \ref{fig:erroneous-labels} demonstrates that the clustering for Mutagenicity focuses on chemical similarity which does not correlate to mutagen similarity.
This means that the concept completeness is likely to be low as demonstrated in \ref{tab:GCN-concepts} though the actual model accuracy may be high.

The average purity is far more erratic with little correlation between the two results which would suggest that the purity score or concept extraction is incorrect.
However, the 13 node cut-off and non-deterministic nature of k-Means\note{citation needed} clustering is likely to be the cause rather than an incorrect implementation.

\fig{GCN-BA-Shapes}{A subset of concepts discovered for BA-Shapes from the best perfomring GCN model. Green nodes highlight the node of interest and pink nodes highlight the neighbourhood used for inference. Each row represents an individual concept.}

\fig{Magister-BA-Shapes}{A subset of the BA Shapes concepts discovered in figures 2, 3 and 5 from \textit{Magister et al.}\cite{magister2021gcexplainer} to demonstrate the full range of labels. Each row represents an individual concept, the same colour system is used as fig. \ref{fig:GCN-BA-Shapes}.}

Figure \ref{fig:GCN-BA-Shapes} presents a subset of the BA Shapes concepts reproduced by the best performing GCN model.
For comparison figure \ref{fig:Magister-BA-Shapes} presents those published in \textit{Magister et al.}
Label 0 is included in both figures to demonstrate that the model does identify the base graph, in this case Barabasi-Albert, but as can be seen this provides little insight into how the model reasons.
The remaining concepts demonstrate the 3 other labels associated with the house motif, as discussed in \Sref{sec:synth}.


As can be seen all the published concepts have an equivalent concept in the reproduced concepts.
The same analysis can be made that the edge attaching the house motif to the base graph is important to the classification of the nodes.
There is also the same distinction in label 1 between a node on the ``inside'' and the ``outside'' where the ``inside'' node has an additional edge attached to the base graph.
This distinction is present in label 2 as well, given that all the nodes present in both the published and reproduced concepts focus on the ``inside'' node.

In both cases, and as expected given the concept purity scores, the concepts related to the motifs are almost completely pure with the exception of label 1 ``inside'' node.
Additionally, all the nodes in the house motif have a unique concept where applicable and this leads to the high completeness score.

Given the visual similarity in concepts between the published and reproduced results I consider the implementation of GCN to be accurate.
Examples of the other synthetic and real datasets is available in Appendix \note{reference when complete}.
\note{There may also be brief comparisons between the figures time permitting.}

\section{Comparison of Accuracy}
\label{sec:comp-acc}
\begin{table}
    \centering
    \begin{tabular}{c|c}
        \textbf{Dataset} & \textbf{Accuracy} \\
        \midrule
        BA Shapes       & 61.4\% $\pm$ 3.3 \\
        BA Grid         & 72.4\% $\pm$ 1.6 \\
        BA Community    & \% $\pm$ \\
        Tree Cycles     & 50.5\% $\pm$ 4.2 \\
        Tree Grid       & 59.2\% $\pm$ 1.6 \\
    \end{tabular}
    \caption{Test accuracy in \% of SGC on each of the synthetic datasets using the hyperparameters in \note{reference when table complete}. As per \textit{Wu et al.}\cite{wu2019simplifying} outliers are removed as defined in \note{reference when complete}.}
    \label{tab:GCN-acc}
\end{table}



Table \ref{tab:SGC-acc} demonstrates the mean accuracies achieved by each of the SGC models using the hyperparameters in \ref{tab:SGC-params}.
As a comparison the accuracies achieved by GCN are presented beside them.
The performanace of SGC is very poor and so random guesses are included as well to demonstrate the limited inference of SGC.

\paragraph{Compared to GCN}
the accuracies achieved by SGC are significantly worse and go against \textit{Wu et al.}'s claim that SGC can match GCN perfomance.
In the cases of the Tree datasets GCN nearly doubles the accuracy and in the case of BA Community GCN is more than $4\times$ as accurate.
Only in the case of BA Grid does SGC achieve a good accuracy in comparison to GCN though even here the difference in accuracy is $27.1$\%.

These poor results suggest that the graph structure inference of SGC is far worse than that of GCN as labels are based only on graph structure.
This is also an explanation for why SGC is able to surpass the accuracy of GCN in the Planetoid\cite{Fey/Lenssen/2019} as these datasets rely heavily on node representations.

\paragraph{Compared to random guesses}
SGC does not perform much better except in the cases of BA Shapes and BA Grid.
In the cases of the Tree datasets this can be attributed to the sparsity of the base graph resulting in less information being aggregated in the pre-computation stage.
Especially in the case of Tree Cycles where the Cycle structure and BST structure share similar sparse connections.

Comparatively the dense BA graph allows for a better distinction between the motifs and the base graph.
This distinction based on degree is attributed to the degree normalisation demonstrated in equation\ref{eq:GCN-as-GNN}.
This advantage is not present in BA Community because of the increase in classes, the need to distinguish the two communities and the possibility of a motif having multiple connections to different base graphs.

These poor results are not due to poor hyperparameter selection as is demonstrated in Appendix \note{reference when complete} and discussed in \Sref{sec:hyperparameters}.
Instead the explanation for the poor performance is due to the lack of proper graph structure inference in SGC.
This is lack of graph structure inference is not in comparison to GCN but rather a fundamental property of SGC.

\section{Comparison of Concepts}
\label{sec:comp-concept}
\begin{table}
    \centering
    \begin{tabular}{c|cccc}
        \multirow{2}{*}{\textbf{Dataset}} &
        \multirow{2}{*}{\textbf{Completeness}} &
        \multicolumn{3}{c}{\textbf{Purity}} \\
        & & \textbf{Max.} & \textbf{Min.} & \textbf{Mean} \\
        \midrule
        BA Shapes       & 0.882 & 0.0 & 5.0 & 1.6 (9)\\
        BA Grid         & 0.843 & 0.0 & 0.0 & 0.0 (2) \\
        BA Community    & 0.264 & 0.0 & 9.0 & 3.5 (6)  \\
        Tree Cycles     & 0.874 & 0.0 & 6.0 & 2.0 (6) \\
        Tree Grid       & 0.890 & 0.0 & 11.0 & 2.1 (8) \\
    \end{tabular}
    \caption{Concept completeness and purity of SGC on each of the synthetic datasets using the hyperparameters in table \ref{tab:SGC-params}. The brackets represent the number of concepts considered for purity, as per \note{reference when complete} these are graphs with less than 13.}
    \label{tab:GCN-concepts}
\end{table}



\section{Extensions}

\subsection{SGC Graph Classification}

\subsection{SGC and GCN Mixed Model}

\subsection{JumpNet style SGC}


\chapter{Conclusion}

%summary

%lessons learned

%future work


\bibliography{references}

\appendix
\chapter{Abbreviations and terms}

\paragraph{AMI}
adjusted mutual information.

\paragraph{BA}
barabasi-albert.

\paragraph{BST}
binary symmetric tree.

\paragraph{GCExplainer}
graph concept explainer --- GNN explainability tool developed by \textit{Magister et al.}\cite{magister2021gcexplainer}.

\paragraph{GCN}
graph convolution network --- GNN proposed by \textit{Kipf et al.}\cite{kipf2016semi} that uses the idea of convolutions on graphs.

\paragraph{GED}
graph edit distance.

\paragraph{GNN}
graph neural network --- a neural network archictecture used for processing graph data.

\paragraph{GRL}
graph representation learning.

\paragraph{JSGC}
jump SGC --- a JKN that combines SGC precomputations of different degrees.

\paragraph{JKN}
jumping knowledge network --- a GNN archictecture proposed by \textit{Xu et al.}\cite{xu2018representation} that can selectively combine different layer aggregations.

\paragraph{MLP}
multi-layer perceptron.

\paragraph{NN}
neural network.

\paragraph{SGC}
simple graph convolution --- a linearised version of GCN.

\paragraph{SGCN}
SGC and GCN --- A mixed model of SGC and GCN starting with SGC precomputations before a GCN.

\paragraph{SGNN}
simplified graph neural network --- GNNs that are linear with no non-linear layers or very few non-linear layers.

\paragraph{t-SNE}
t-distributed stochastic neighbour embedding.

\paragraph{linear GNN}
A GNN archictecture which includes no non-linear layers and is instead a series of different graph or matrix operations.

\paragraph{quasi-linear GNN}
A GNN archictecture which includes one non-linear layer separating linear graph representation learning from classification.


\chapter{Hyperparameters}
\section{Hyperparameters}
\begin{table}
    \centering
    \begin{tabular}{c|ccc}
        \textbf{Dataset} &
        \textbf{Layers} &
        \textbf{Learning rate} &
        \textbf{Epochs} \\
        \midrule
        BA Shapes       & 3 $\times$ 20 & 0.001 & 3000 \\
        BA Grid         & 3 $\times$ 20 & 0.001 & 3000 \\
        BA Community    & 6 $\times$ 50 & 0.001 & 6000 \\
        Tree Cycles     & 3 $\times$ 50 & 0.001 & 7000 \\
        Tree Grid       & 7 $\times$ 20 & 0.001 & 10000 \\
        \midrule \\
        REDDIT BINARY   & 4 $\times$ 40 & 0.005 & 3000 \\
        Mutagenicity    & 4 $\times$ 30 & 0.005 & 10000 \\
    \end{tabular}
    \caption{GCN hyperparameters from table 15 in \textit{Magister et al.}\cite{magister2021gcexplainer}. All models use \texttt{ReLU}\note{citation needed?} activation functions between layers with a single linear layer classifier with \texttt{softmax}\note{citation needed?}.}
    \label{tab:GCN-params}
\end{table}


\begin{table}
    \centering
    \begin{tabular}{c|cccc}
        \textbf{Dataset} &
        \textbf{Degree} &
        \textbf{Learning rate} &
        \textbf{Weight decay} &
        \textbf{Epochs} \\
        \midrule
        Cora        & 2 & 0.002 & 0.04 & 100 \\
        Citeseer    & 2 & 0.004 & 0.08 & 100 \\
        PubMed      & 2 & 0.0005 & 0.25 & 100 \\
    \end{tabular}
    \caption{SGC hyperparameters using \texttt{hyperopt}\note{citation needed} as proposed by \textit{Wu et al.}\cite{wu2019simplifying}. All models use \texttt{softmax}\note{citation needed?}.}
    \label{tab:SGC-reproduction-params}
\end{table}


\begin{table}
    \centering
    \begin{tabular}{c|cccc}
        \textbf{Dataset} &
        \textbf{Degree} &
        \textbf{Learning rate} &
        \textbf{Weight decay} &
        \textbf{Epochs} \\
        \midrule
        BA Shapes       & 3 & 0.01 & 0.1 & 3000 \\
        BA Grid         & 3 & 0.01 & 0.1 & 3000 \\
        BA Community    & 6 & 0.001 & 0.1 & 6000 \\
        Tree Cycles     & 3 & 0.01 & 1.0 & 7000 \\
        Tree Grid       & 7 & 0.01 & 0.01 & 10000 \\
    \end{tabular}
    \caption{SGC hyperparameters based on the corresponding GCN hyperparameters in \ref{tab:GCN-params} and using the results of the hyperparameter search in \note{reference when added}. All models use \texttt{softmax}\note{citation needed?}.}
    \label{tab:SGC-params}
\end{table}


\begin{table}
    \centering
    \begin{tabular}{c|cc}
        \textbf{Dataset} &
        \textbf{Clusters (k)} &
        \textbf{Receptive field (n)} \\
        \midrule
        BA Shapes       & 10 & 2 \\
        BA Grid         & 10 & 3 \\
        BA Community    & 30 & 2 \\
        Tree Cycles     & 10 & 3 \\
        Tree Grid       & 10 & 3 \\
        \midrule
        REDDIT BINARY   & 20 & 1 \\
        Mutagenicity    & 30 & 3 \\
    \end{tabular}
    \caption{Concept extraction parameters used in \textit{Magister et al.}\cite{magister2021gcexplainer}.}
    \label{tab:GCN-concept-params}
\end{table}


\begin{table}[h]
    \centering
    \captionsetup{width=.9\textwidth}
    \begin{tabular}{c|cccc}
        \textbf{Dataset} &
        \textbf{Degree} &
        \textbf{Learning rate} &
        \textbf{Weight decay} &
        \textbf{Epochs} \\
        \midrule
        BA Shapes       & 3 & 0.001 & 0.01 & 1200 \\
        BA Community    & 6 & 0.001 & 0.01 & 1400 \\
    \end{tabular}
    \caption{JSGC hyperparameters based on the corresponding SGC hyperparameters in \ref{tab:SGC-params} and using the results of the hyperparameter searchs in figure \ref{fig:JSGC-surfaces}. All models use \texttt{softmax} and \texttt{ReLU} between the aggregation of jumping knowledge and the classifier.}
    \label{tab:JSGC-params}
\end{table}



\section{Hyperparameter surfaces}
\label{sec:surfaces}

\begin{figure}
    \centering
    \includegraphics[width=0.45\textwidth]{figures/SGC-BA-Community-surface}
    \includegraphics[width=0.45\textwidth]{figures/SGC-BA-Grid-surface}
    \includegraphics[width=0.45\textwidth]{figures/SGC-BA-Shapes-surface}
    \includegraphics[width=0.45\textwidth]{figures/SGC-Tree-Cycles-surface}
    \includegraphics[width=0.45\textwidth]{figures/SGC-Tree-Grid-surface}
    \caption{SGC hyperparameter surfaces for learning rates in $\{0.01, 0.001, 0.0001\}$ and weight decay constants in $\{1, 0.1, 0.01\}$. In the case of Tree Cycles enough of an improvement is seen to warrant the additional weight decay constant $10$.}
    \label{fig:SGC-surfaces}
\end{figure}

\begin{figure}
    \centering
    \includegraphics[width=0.45\textwidth]{figures/JSGC-BA-Community-surface}
    \includegraphics[width=0.45\textwidth]{figures/JSGC-BA-Shapes-surface}
    \caption{JSGC hyperparameter surfaces for learning rates in $\{0.01, 0.001, 0.0001\}$ and weight decay constants in $\{1, 0.1, 0.01\}$. In the case of BA Shapes enough of an improvement is seen to warrant the additional weight decay constant $10$.}
    \label{fig:JSGC-surfaces}
\end{figure}


\chapter{Concepts}
\label{app:concepts}

\section{SGC concepts}

\fig{BA-Grid}{Comparison of SGC and GCN concepts for BA Grid further demonstrating the poor graph structural inference of SGC. The numbered concepts are a subset of SGC concepts and the lettered concepts are GCN concepts. The colour scheme is the same as fig. \ref{fig:SGC-BA-Shapes}}
\fig{Tree-Cycles}{Comparison of SGC and GCN concepts for Tree Cycles further demonstrating the poor graph structural inference of SGC. The numbered concepts are a subset of SGC concepts and the lettered concepts are GCN concepts. The colour scheme is the same as fig. \ref{fig:SGC-BA-Shapes}}
\fig{Tree-Grid}{Comparison of SGC and GCN concepts for Tree Grid further demonstrating the poor graph structural inference of SGC. The numbered concepts are a subset of SGC concepts and the lettered concepts are GCN concepts. The colour scheme is the same as fig. \ref{fig:SGC-BA-Shapes}}

Figure \ref{fig:BA-Grid} represents a sample of concepts from SGC and GCN on BA Grid to demonstrate the differences.
SGC is able to discern between the Barabasi-Albert graph as shown by concepts 1, 2 and 3, but in comparison to GCN these are not as detailed.
Concepts 1 and 2 dmeonstrate an inconsistent node of interest within the grid structure compared to concepts A, B and C.
Concepts A, B and C demonstrate that the central, edge and connecting nodes are distinct and important to the classification used by GCN.
This demonstrates a higher level of graph structure awareness from GCN compared to SGC.

Figure \ref{fig:Tree-Cycles} represents a sample of concepts from SGC and GCN on Tree Cycles to demonstrate the differences.
Of the graph concepts SGC matches GCN the closest on Tree Cycles even though this is one of the poorly performing datasets as shown in table \ref{tab:SGC-acc}.
Concepts 1 and 2 match concepts A and B in regards to position on the motif and general shape.
Concept C has limited presence in the concepts extracted from SGC and in comparison to concept D concept 3 shows limited awareness of the full structure of the BST.

Figure \ref{fig:Tree-Grid} represents a sample of concepts from SGC and GCN on Tree Grid to demonstrate the differences.
In comparison to SGC in figure \ref{fig:BA-Grid} the instances of the grid are more pronounced however in this case there a limited and less pure.
Concept 1 shows the major problem that SGC faces with tree based synthetic graphs as it cannot discern the tree structure from the grid (or circle) motif.
In this case concept 3 shows better structural awareness of the BST.
Concepts A to D demonstrate a similar structural awareness to GCN in figure \ref{fig:BA-Grid}.
The only deviation is in concept D as the base graph is different.

\section{SGCN concepts}

\fig{SGCN-BA-Shapes}{Subset of SGCN concepts for BA Shapes demonstrating the improved graph concepts with aspects of both SGC and GCN. Concepts are extracted from the final GCN layer only. The colour scheme is the same as fig. \ref{fig:SGC-BA-Shapes}}

Figure \ref{fig:SGCN-BA-Shapes} represents a sample of concepts from SGCN on BA Shapes to demonstrate the differences.
The concepts are a great improvement on those presented in figure \ref{fig:SGC-BA-Shapes} and match those produced in figure \ref{fig:GCN-BA-Shapes}.
Note however in concept 2 that there is no distinction between the ``inside'' and ``outside''\footnote{See \Sref{}} node that is demonstrated by either SGC or GCN.
This oddity may be due to GCN relearning the node representations but unable to properly distinguish between the ``inside'' and ``outside'' nodes presented by SGC.
Given the lower accuracy of SGCN compared to GCN demonstrated in table \ref{tab:SGCN-acc} suggests that the distinction may be important.

\section{JSGC concepts}

\fig{JSGC-BA-Shapes}{Subset of JSGC for BA Shapes further demonstrating the improved graph structural awareness of JSGC. The concepts are chosen to match those in fig. \ref{fig:GCN-BA-Shapes} to demonstrate that the graph structure awareness is the same. The colour scheme is the same as fig. \ref{fig:GCN-BA-Shapes}}

Figure \ref{fig:JSGC-BA-Shapes} represents a sample of concepts from JSGC on BA Shapes to demonstrate the differences.
The subset of concepts directly map to those discovered in \ref{fig:GCN-BA-Shapes}.
In the case of concept 2 in figure \ref{fig:JSGC-BA-Shapes} the probabilistic Barabasi-Albert graph means that they will not match concept 2 in figure \ref{fig:GCN-BA-Shapes}.
This demonstrates that JSGC has almost the same graph structure awareness that GCN does.
As discussed in \Sref{sec:Jump-SGC} and shown in table \ref{tab:JSGC-acc} there are still lacking elements.
It is hypothesised that this is due to the limited classifier in JSGC where GCN acts an MLP as well as carrying out GRL.

\chapter{Proofs}
\section{Equivariance}
\label{app:perm}

\paragraph{Lemma.}
If $\bm{\phi}$ is permutation invariant in $\mX_{\sN_i}$ then $F(\mX, \mA)$ is equivariant.

\paragraph{Proof.}
Let $\pi$ be a permutation of $\sV$, the set of nodes, and $\mP$ the corresponding permutation matrix.
Given $\mX = \begin{bmatrix}\vx_1 \\ \vdots \\ \vx_n\end{bmatrix}$, $\mP\mX = \begin{bmatrix}\vx_{\pi(1)} \\ \vdots \\ \vx_{\pi(N)}\end{bmatrix}$ noting that each node $v_i$ has the same neighbours, $\sN_i$, however the feature matrix of the neighbours is now $\mP\mX_{\sN_i}$. Therefore $F(\mX, \mA) = \begin{bmatrix}\bm{\phi}(\vx_{\pi(1)}, \mP\mX_{\sN_{\pi(1)}}) \\ \vdots \\ \bm{\phi}(\vx_{\pi(N)}, \mP\mX_{\sN_{\pi(N)}})\end{bmatrix}$. As $\bm{\phi}$ is permutation invariant in $\mX_{\sN_i}$, $\bm{\phi}(\vx_{\pi(i)}, \mP\mX_{\sN_{\pi(i)}}) = \bm{\phi}(\vx_{\pi(i)}, \mX_{\sN_{\pi(i)}})$.

Thus 
\begin{align*}
    F(\mP\mX, \mP\mA\mP^T) &= \begin{bmatrix}\bm{\phi}(\vx_{\pi(1)}, \mP\mX_{\sN_{\pi(1)}}) \\ \vdots \\ \bm{\phi}(\vx_{\pi(N)}, \mP\mX_{\sN_{\pi(N)}})\end{bmatrix} \\ 
        &= \begin{bmatrix}\bm{\phi}(\vx_{\pi(1)}, \mX_{\sN_{\pi(1)}}) \\ \vdots \\ \bm{\phi}(\vx_{\pi(N)}, \mX_{\sN_{\pi(N)}})\end{bmatrix} \\
        &= \mP\begin{bmatrix}\bm{\phi}(\vx_1, \mX_{\sN_1}) \\ \vdots \\ \bm{\phi}(\vx_N, \mX_{\sN_N})\end{bmatrix} \\
        &= \mP F(\mX, \mA).
\end{align*}

\section{Orthonormal eigenvectors and the hadamard product}
\label{app:orth}


\chapter{Graph Convolution Network}

\section{Chebyshev polynomial filter}
\label{app:chebyshev}

Chebyshev polynomials are recursively defined as the following

\begin{align}
    T_k(x) &= 2xT_{k-1}(x) - T_{k-2}(x), \\
    T_0(x) &= 1, \\
    T_1(x) &= x.
\end{align}

Rather than define $\hat\mF_\theta$ as a power series on the eigenvalues of the graph Laplacian, $\bm{\Theta}$ it can be defined as a Chebyshev polynomial

\begin{equation}
    \hat\mF_\theta = \sum_{j=0}^k\theta_jT_j(\widetilde{\bm{\Lambda}}).
\end{equation}

where $\widetilde{\bm{\Lambda}} = \frac2{\lambda_{max}}\bm{\Lambda} - \mI$ and $\lambda_{max}$ is the largest eigenvalue. This keeps the values of $\bm{\Lambda}$ in the range $(-1, 1]$.

Looking at equation \ref{eq:GCN-comp} this allows the new formulation of the graph convolution as 

\begin{equation}
    \label{eq:chebnet}
    f_\theta \star_\sG \vh_i = \sum_{j=0}^k \theta_jT_j(\widetilde{\mL})\vh_i.
\end{equation}

where $\widetilde{\mL} = \frac2{\lambda_{max}}\mL - \mI$. It can be assumed that $\lambda_{max} = 2$ as during training the parameters will adapt to any scaling on the filter. This means that $\widetilde{\mL} = \mL - \mI$ Therefore the first order approximation of the Chebyshev polynomial filter in equation \ref{eq:chebnet} is 

\begin{equation}
    (\theta_1\mI - \theta_2(\mL - \mI))\vh_i
\end{equation}

%\section{Graph laplacian}
%\label{app:laplacian}
%
%\error{Include discussion of what the laplacian is, the eigendecomposition, and the normalised laplacian}


\chapter{Datasets}
\label{app:datasets}

\fig{inductive-vs-transductive}{(a) demonstrates transductive learning where bold outlines means that the node features are accessible to the model during training. The solid lines represent the node labels used during training and the dashed lines represent nodes used during testing. (b) demonstrates inductive learning the same system is used as with (a) however now the nodes during testing are not bold and thus not seen during training. As can be seen in (a) removing the dashed nodes would result in all training nodes being disconnected which is not the case in (b).}

\section{Transductive datasets}
These datasets are such that during training the node representations of nodes outside of the training set are ``seen'' by the model but importantly the model does not ``see'' the labels for these nodes.
The model is tested on the labels of these ``unseen'' nodes without access to the training sets labels.
This is because the training, validation and test sets are sampled from the nodes of a single graph, if only these nodes where given to the model the structure of the graph is lost and a large proportion of the nodes will no longer be connected.
This dataset forces the model to transfer knowledge about existing (representation, label) pairs to unlabelled nodes.
This concept is present in figure \ref{fig:inductive-vs-transductive}(a).

\section{Inductive datasets}
These datasets are such that the nodes ``seen'' during training, validation and testing are completely disjoint.
This is generally the case in graph classification tasks where each data point is a unique graph with a classification thus the test set is a new set of graphs with completely unseen nodes.
This dataset forces the model to induce information of relationships between nodes and optimal representations.
This concept is present in figure \ref{fig:inductive-vs-transductive}(b).


\chapter{Proposal}
\label{ch:proposal}
\section{Description}

\subsection{Introduction}

Within the area of geometric deep learning there have been recent ablation studies looking into the effectiveness of Graph Neural Networks (GNNs). The majority of these studies question the effectiveness of the deep neural network approach of multiple layers separated by non-linear function passes when working with geometric datasets (graphs). \cite{wu2019simplifying} introduce a new approach, Simplified Graph Convolution (SGC), which remove these non-linear functions from the network. This reduces the problem to a pre-computation on the graph adjacency matrix and a simple linear regression using a single weight matrix. The pre-computation on the graph adjacency matrix encodes information about message passing between nodes in the graph.  \cite{chanpuriya2022simplified} introduce further variations on SGC that use the same underlying concept of a pre-computation but deal with the parameters differently allowing for more complex associations. In both cases the results show that removing the non-linearity does not hinder the performance of the network and can in fact improve performance.

Similarly, the has been a lot of interest into explainable artificial intelligence (XAI) to move away from the black box nature of AI models. There exists multiple methods within this field of machine learning and I will specifically focus on the idea of Concepts. Concepts focuses on relating specific outputs of a model to subspaces within its input space, this gives an indication of what the model is using within the input space to infer the given output. The collections of these subspaces are what are known as concepts. This approaches allows a human actor to get a better understanding of the model's inference as they can compare their own intuition of the input to the concepts the model uses to produce the given result.
\cite{magister2021gcexplainer} introduce GCExplainer which adapts prior techniques to extract high-level concepts from GNNs. The paper focuses on extracting concepts from a Graph Convolutional Network (GCN, \cite{kipf2016semi}) model.

Both SGC and GCExplainer are research papers but both including detailed sections on experiment setup and available github repositories. In the case of GCExplainer the author of the paper is one of my supervisors and can therefore clear up any setup details and help with common pitfalls in the training process.

\subsection{Extensions}

Further extensions on the core goal will look at the concepts extracted from variations on SGC with potential new approaches outside of existing literature being implemented. Approaches outside of literature include mixing SGC "layers" within GCN layers, allowing for non-linearity to be added later in the forward pass. Equally using multiple SGC steps and using Jump Knowledge \cite{xu2018representation} to help combat the common problem of over-smoothing in GNNs. Further extensions may be found based on results during the project.

These extensions can also look into how accurate the new models are outside of the concept metrics and provide further comparison between accuracy and concept purity \& completeness.

\subsection{Substance and Structure}

With the rise in popularity of simplified GNNs due to their simple nature and effective representational learning it is important to evaluate how they fit into explainability frameworks. This project therefore looks to extract concepts from SGC trained on the same datasets using in \cite{magister2021gcexplainer} to compare against GCN. This work will provide useful insight into how these simplified networks operate as well as providing further comparison between previous GNN techniques (GCN) and new simplified GNN techniques (SGC). Comparison between the two GNN techniques will focus on concept purity and completeness as described in \cite{magister2021gcexplainer}.

This is broken down into reproducing both papers using the setup and structure described in each to compare the results that I achieve with those published in each paper. The code used for both of these can then be combined to extract concepts from the SGC model. This will allow comparison between a simplified GNN (SGC) and a standard GNN (GCN). Further extensions can then be carried out based on these results to further investigate different GNN approaches in regard to concept extraction. 

\section{Starting Point}

I have worked with Pytorch and the extension for geometric deep learning, Pytorch Geometric, over the summer culminating in a research paper. During this summer project I worked with building and testing graph datasets, building new GNN models, and testing/comparing GNN models from previous models to the new models. During the project I also briefly learnt about and used convolutional neural networks and transformers. I do not have experience with XAI or the area of Concepts.

No code has been written for this project beyond the code that will be used within the Pytorch and Pytorch Geometric libraries.

\section{Success Criteria}

The project will be deemed a success if I 

\begin{itemize}[noitemsep]
    \item Implement SGC and extract the concepts used for each of the datasets
    \item Implement GCN and extract the concepts to use as a baseline
    \item Compare the concepts between SGC and GCN using the metrics of concept completeness and concept purity
\end{itemize}

\section{Work Plan}

\subsection{Interval 1: 13/10 - 26/10}

\textit{Preparatory Work:} Download datasets (both SGC and GCExplainer), preliminary tests on datasets to check all correct, setup up environment (required libraries are up-to-date etc.), test training on local machine and servers to determine extra resources. Read and annotate the three papers being implemented.

Start work on implementing Simplified Graph Convolution (SGC).

\subsection{Interval 2: 27/10 - 9/11}

Complete implementation of SGC and train on SGC paper datasets, compare against papers results to confirm implementation is correct. Train SGC on GCExplainer datasets.

\textbf{Milestone:} Trained SGC model.

\textit{Category Theory 1 Deadline: 4/11 (25\%)}

\subsection{Interval 3: 10/11 - 23/11}

Implement Graph Convolution Network (GCN, from GCExplainer) and train on GCExplainer datasets.

\subsection{Interval 4: 24/11 - 7/12}

Implement GCExplainer concept extraction on GCN and compare concept metrics to confirm implementation is correct.

\textbf{Milestone:} Working GCExplainer.

\textit{Category Theory 2 Deadline: 2/12 (75\%)}

\subsection{Interval 5: 8/12 - 21/12}

Carry out concept extraction on the trained SGC model(s) and produce concept metrics for SGC model(s).

\textbf{Milestone:} SGC concept metrics (\textbf{Core Project Complete})

\subsection{Interval 6: 22/12 - 4/1}

\textit{Buffer.} Christmas and New Year

\subsection{Interval 7: 5/1 - 18/1}
\label{interval:extension}

\textit{Extension:} Research (read papers and annotate) one of the following techniques

\begin{itemize}[noitemsep]
    \item Mixing SGC and GCN layers
    \item Adding knowledge jumps
    \item Other simplified graph convolutions
\end{itemize}

\subsection{Interval 8: 19/1 - 1/2}

\textit{Extension:} Implement and train the chosen technique from \Cref{interval:extension} on the GCExplainer Datasets.

\subsection{Interval 9: 2/2 - 15/2}

\textit{Extension:} Carry out GCExplinaer concept extraction on the trained model from \Cref{interval:extension}.

\textbf{Deadline:} Progress Report 3/2

\subsection{Interval 10: 16/2 - 1/3}

\textit{Write Dissertation:} Introduction and Implementation chapters, and share with supervisors.

\textit{Deep Neural Networks 1 Deadline: 17/02 (30\%)}

\subsection{Interval 11: 2/3 - 15/3}

\textit{Buffer.} If not required and extension runs smoothly carry out a second technique from \Cref{interval:extension}.

\subsection{Interval 12: 16/3 - 29/3}

\textit{Write Dissertation:} Evaluation of different approaches and their concept metrics, and share with supervisors.

\textbf{Milestone:} First draft

\subsection{Interval 13: 30/3 - 12/4}

\textit{Write Dissertation:} Respond to feedback on first draft and share with supervisors.

\textbf{Milestone:} Second draft

\textit{Deep Neural Networks 2 Deadline: 17/03 (70\%)}

\subsection{Interval 14: 13/4 - 27/4}

\textit{Write Dissertation:} Respond to feedback on second draft and share with supervisors.

\textbf{Milestone:} Final draft

\textbf{Deadline:} Dissertation submission 12/5

\section{Supervisors}

\begin{itemize}[noitemsep]
    \item Lucie Charlotte Magister
    \item Pietro Babiero
    \item Pietro Lio
\end{itemize}

I will have joint weekly meetings at 5pm on Wednesday will all supervisors (barring temporary unavailability) to report on the progress of the project and help with any problems that have occurred in the current interval. 

\section{Resource Declaration}

My own machine (AMD Ryzen 7 5700U with Radeon Graphics (16) @ 1.8GHz, 16GB RAM, and AMD ATI Lucienne) for the primary implementation of the project and remote work on any provided servers used. The majority of the project should be completed on this machine. I will also use git version control storing a remote repository on a private github repository in the eventuality of my own machine malfunctioning. Regular remote pushes at the end of the day or major project milestones will be carried out. I accept full responsibility for this machine and I have made contingency plans to protect myself against hardware and/or software failure.

High performance cluster (HPC) as intensive training may be required. The required hours will remain within the SL3 bracket so no funding for SL2 will be required. 

Local CL server access (idun, heimdall, etc) for less costly iterations and testing of GPU models before utilising HPC if my own machine is not sufficient. I am able to acquire kerberos tickets to use these services.

Storage space available on my own machine will be sufficienet for the datasets used. Though space on local CL and HPC servers would be required if I am using these for model training.

I will require other software packages, PyTorch and PyTorch Geometric, to help with the framework of building, testing and training my models.


%\bibliography{references}

%\end{document}


\end{document}
