\chapter{Implementation}

\section{Datasets}
\label{sec:datasets-imp}
\Sref{sec:datasets-theory} presents a number of datasets that cover a range of different GNN training styles and motifs.
The inductive datasets discussed in \Sref{sec:RWD} are already packaged and available through \texttt{PyTorch Geometric}\cite{Fey/Lenssen/2019} as well as the Planetoid\cite{planetoid}\cite{citation} datasets.

\paragraph{Synthetic datasets}
The synthetic datasets discussed in \Sref{sec:synth} are proposed by \textit{Ying et al.}\cite{ying2019gnnexplainer} are available pre-built in their github and \texttt{PyTorch Geometric} provide limited versions of a small subset I decide to implement them myself.
This allows for more rigorous testing of both the datasets and the SGC precomputation later described in \Sref{sec:testing-imp}.

The motifs described in \Sref{sec:synth} are simple enough to be hard-coded, each motif is therefore an adjacency matrix (in coordinate form) and a label vector.
When generating the dataset the base graph adjacency is generated with a label vector assigning every node to class 0.
The Barabasi-Albert graph is generated probabilistically adding one node at a time and connecting each node to the existing graph based on the degree of the nodes currently in the graph.
Then for each motif required the hard-coded tensors are adjusted and appended to the on going adjacency matrix and label vector.
This adjustment includes adding an additional bidirectional connection between a random node in the base graph and a predetermined node of the motif.

Though the datasets are very simple they need to be compliant with \texttt{PyTorch Geometric} to be used correctly during training.
\texttt{PyTorch Geometric} provides an \texttt{InMemoryDataset} abstract class for datasets that are small enough to be generated and stored within RAM.
As these are transductive datasets as described in \Sref{sec:datasets-theory} a single graph \texttt{Data} object is created that is then accessed through a dataloader.

The \texttt{Data} object includes train and test split masks randomly assigning nodes to achieve an 80:20 ratio.
This random assignment is based on the seed value for the experiment discussed in \Sref{sec:reproducibility}.
This randomness also applies to the generation of the graph creating significantly different graphs each generation.
This means that nodes for one seed value are effectively unseen for another seed value.

\paragraph{SGC}
The majority of SGC is the pre-computation
\begin{equation}
    \label{eq:pre-comp}
    \mS^k\mX
\end{equation}
of equation \ref{eq:SGC} where $\mS$ is the filter defined in equations \ref{eq:op} and \ref{eq:norm}.
As $\mS$ is defined in terms of the adjacency matrix which does not change during training equation \ref{eq:pre-comp} can be applied before training.
This process is achieved with three functions that calculated the normalised filter $\mS$ by computing the required matrices and multiplications, convert the resultant matrix from coordinate form to sparse tensor form and then successively apply the filter to the node representations.

This process results in a new feature matrix which can be substituted for the \texttt{Data} object used by the synthetic datasets.
This means the SGC model does not need any adjacency matrix and can instead be implemented as a classifier.

For the sake of consistency and to aid the concept evaluation functions in \Sref{sec:concepts} the adjacency matrix is keep.
A new \texttt{Data} object is created copying the labels, data splits and adjacency of the input graph but with the pre-computation matrix as the feature matrix.
This also allows for implementation of the SGC and GCN mixed model proposed in \Sref{sec:SGCN}.

\section{Models}
\label{sec:models}

\texttt{PyTorch Geometric} provides an abstract \texttt{MessagePassing} class which defines a generic GNN layer.
Beyond the standard \texttt{forward} method of \texttt{PyTorch}\note{citation needed} an addition \texttt{message\_passing} method is also present.
This method is called during the \texttt{forward} method providing the node representations and associated neighbourhood nodes.
This method allows the implementation of the function
\begin{equation}
    \mH^{(l+1)} = \mS\mH^{(l)}\bm{\Theta}^{(l)},
\end{equation}
described in \ref{sec:GCN} as equation \ref{eq:GCN1}.
During the \texttt{message\_passing} method the normalised adjacency $\mS$ is applied to the node representations $\mH^{(l)}$.
Then a linear layer is applied to the resulting representations in the \texttt{forward} method.

These GCN layers can then be combined with ReLU non-linearity to form a GCN model.
The GCN model additionally includes a single linear layer for classification.
In comparison SGC needs only to be a classifier as explained in \Sref{sec:datasets-imp} and so contains a single linear layer.

\section{Machine Learning Pipeline}

The ML pipeline is the infrastructure that links the datasets, pre-computation, dataloaders, and models together to train and evaluate the models.
This allows for specific experiment configurations to be run changing each component individually or together.
As the main aspect of this project is concept extraction, evaluation, and interpretation I use \texttt{PyTorch Lightning}\note{citation needed} as the main component of the pipeline.

\texttt{PyTorch Lightning} provides \texttt{Lightning Modules} that act as wrappers around standard \texttt{PyTorch} (and therefore \texttt{PyTorch Geometric}) modules and carry out all the required training, validation and testing loops.
Unfortunately, for the purposes of the core project and extensions a single wrapper is not possible.
This is because the wrappers need to behave differently for SGC and GCN as GCN needs additional adjacency information.
Additionally the real-world graph classification datasets require one-hot encoding when calculating loss whereas the synthetic datasets do not.
Therefore the project implements 5 different wrappers to accomodate these differences.

The datasets, pre-computation and dataloaders discussed in \Sref{sec:datasets-imp}, the models described in \Sref{sec:models} and the wrappers are then instantiated, run and results saved in a single \texttt{main.py} file.
To run a single experiment the specific model build, dataset, seed, save destinations, etc. need to be passed to this function.
To reduce this overhead multiple bash scripts can be dynamically created to run single quick experiments or full experiments which test the model on prechosen random seeds.

\subsection{Reproducibility}
\label{sec:reproducibility}
It is important that the results that are achieved by the models, especially in regards to model accuracy, can be replicated later.
This is both for the results presented in this dissertation but also for results found during experimentation in the extension phase.

To achieve this every single experiment must set a seed value which is saved along with the results.
This way when running the same configuration again using the same seed the expected outcome should be the same within very small bounds.

Furthermore, the results achieved should be linked to specific model build that identifies all the hyperparameters used.
The model build specifies every aspect of the model so that the exact same model can be replicated in a completely different ML pipeline.
Thus the build files need to be human readable and easily expandable.

To achieve these two requirements I use YAML which mimics python in its syntax and is therefore very readable but also links well with the rest of the pipeline.
Each build has a unique YAML file with the filename following the convention \texttt{<model>.<dataloader>.<dataset>.<version>.yml}.
Storing the YAML filename, seed and timestamp with every set of results allows for the reproduction of almost every result that was achieved throughout the project.

\subsection{Experimentation}
The infrastructure used in \Sref{sec:reproducibility} also add in experimentation but linking interesting or desired results to the specific configuration that produced them.
This is particularly desirable during the extension phase where the exact method that will produce the best results is unknown.

Beyond quick iteration in models during the exploratory extensions finding the optimal hyperparameters is essential.
Incorrect hyperparameters can lead to artificially low performance and thus invalid the interpretation of the models based on results.
I therefore implement \texttt{sweep.py} which takes a set of different hyperparameters values to test for each hyperparameter and runs a short evaluation of the performance.
To properly evaluate models it is important that this evaluation does not use nodes that are in the final test set.
As discussed in \Sref{sec:datasets-imp} choosing different seeds to the final run effectively means that nodes in the prechosen random seeds test set are always unseen.

\section{Concept Extraction and Evaluation}
\label{sec:concepts}

\subsection{Extraction}
\label{sec:extraction}

\subsection{Evaluation}
\label{sec:concept-eval}

\subsection{Visualisation}
\label{sec:vis}

\section{Testing}
\label{sec:testing-imp}

\section{Extensions}
After completing the core project my extensions focus on improving the low performance demonstrated by SGC in \Sref{sec:comp-acc}.
The motivation for some of my extensions develops from the results and analysis of the core project however the implementation details are non-trivial.
This section outlines the three extensions completed, the motivation for each and the implementation details.
The evaluation of my extensions is presented in \Sref{sec:extension-eval}.

\subsection{SGC graph classification}
\paragraph{Motivation}
The datasets present in \textit{Magister et al.}\cite{magister2021gcexplainer} include 2 real world datasets that focus on graph classification.
As discussed in \Sref{sec:datasets-theory} these graph classification datasets are also inductive rather than transductive which provides another test of the capabilities of SGC.
Furthermore, though \Sref{sec:comp-acc} suggests SGC performs poorly, this could be because of the synthetic nature of the datasets.

\paragraph{Prior work}
\textit{Wu et al.}\cite{wu2019simplifying} discuss graph classification for SGC suggesting that it can substitute GCN in a deep graph convolutional neural network as proposed by \note{name needed}\note{citation needed}.
However, \textit{Magister et al.} carry out graph classification using GCN in a different way.
Instead they utilise pooling on the graph node representations from a GCN on each graph in the dataset.
The label of the entire graph can then be inferred from this single representation.
As this latter method is easier to implement and allows for a fairer comparison between SGC and the results achieved by \textit{Magister et al.} this approach is chosen.
The resulting model is identical to the standard SGC model with addition of a pooling layer before the classifier.

\paragraph{Implementation}
The only problem posed is concept extraction as this is done on the node level, as suggested by \textit{Magister et al.}, rather than the graph level.
Each graph has a single label with no labels for the node, therefore the method proposed in \Sref{sec:concept-eval}, does not work.
This is overcome by broadcasting the graph label to each of the nodes in the graph.
To achieve this additional information about the batch identifier of each node and each label is used to match labels to nodes.

This allows the graphs to be combined into a disconnected forest of graphs and clustering can be carried out on this forest.
Calculation of concept scores and the visualisation of concept therefore remains the same as that presented in \Sref{sec:concept-eval} and \Sref{sec:vis}.

\subsection{SGC and GCN mixed model}
\label{sec:SGCN}
\error{Make sure this makes sense to flow.}
\paragraph{Motivation}
The derivation in \Sref{sec:SGC} shows that SGC is directly derived from GCN using the same graph filter.
Therefore, though the two models use this filter differently they should be interchangeable to some extent.

Futhermore the benefit of SGC is to reduce the complexity of GCN and the cost of training by pre-computing the graph operation.
Even though for the very small datasets in \Sref{sec:datasets} the improvement on cost and reduction in parameters is not significant larger datasets may create a larger benefit.

However, due to the low accuracy of SGC seen in \Sref{sec:comp-acc} some properties of GCN must required for high accuracy 
A mixture of both should therefore yield a high accuracy model with fast pre-computation.
This model will be referred to as SGCN.

\paragraph{Implementation}
\note{A more detailed explanation is now possible}
For simplicity and to utilise the power of pre-computation SGCN starts with SGC layers and then transitions to GCN layers.
For a fairer comparison the total number of layers must remain the same as the corresponding SGC and GCN layers.
This reduces the problem to finding where SGC and GCN agree the most in regards to their concepts.

This is achieved using mutual information between the two models by using the probability of a node appearing in a specific cluster from either model.
This gives a measure for the dependence of the models clusters and therefore how similar the models are.
However, this does not take into account random chance of two nodes appearing in the same cluster.
Therefore the mutual information is adjusted for this chance resulting in a number in the range $[0, 1]$ where $1$ is identical.
This is known as \emph{adjusted mutual information}(AMI).

To better visualise these results the dimensionality of the activations from the models are reduced using \emph{t-distributed stochastic neighbor embedding} (t-SNE)\note{citation needed} into 2 dimensions.
This clusters similar representations together and keeps different representations apart which acts as a visual proxy to viewing all the concepts of a model.
These can then be compared to see why a specific layer has the most mutual information and how the join effects the resulting model.

\subsection{Jumping knowledge SGC}
\error{Make sure this makes sense to flow.}
\paragraph{Motivation}
\Sref{sec:concept-analysis} discusses the limited influence SGC has on node representations during message passing.
An SGC model, of degree $k$, can only infer graph structure from the aggregated node representations of all neighbouring nodes within $k$ hops.
In comparison GCN is able to manipulate node representations between graph convolutions and can therefore learn to amplify or dampen differences between node representations which SGC cannot do.
Therefore a potential reason for the low accuracy (seen in \Sref{sec:comp-acc}) and poor graph structure awareness (seen in \Sref{sec:comp-concept}) may not be due to the linearity.

For these reasons I propose \emph{jump-SGC}(JSGC) which provides the classifier with node representations from each degree of the pre-computation.
This larger node representation is reduced before the classifier to maintain the same parameter size of the classifier.
This allows for JSGC to effectively manipulate the node representations though these manipulations do not have an impact on the successive application of the graph filter.
This idea mimics \emph{jumping knowledge networks} (JKNs) proposed by \textit{Xu et al.}\cite{xu2018representation} hence the name ``jump''.

\paragraph{Prior work and Implementation}
\note{A more detailed discussion is possible.}
\textit{Xu et al.}\cite{xu2018representation} identify the drawbacks of node aggregation in accurately representing a nodes neighbourhood.
It specifically identifies the effect on graph structure awareness this has making the method ideal for SGC.
The motivation for JCNs is the node aggregation methods used resulted in neighbourhood influence similar to a random walk rather than a uniform influence.
As a solution they propose aggregating the the node representations after successive neighbourhood aggregation layers together.
Three main methods of aggregation are proposed but given the small size of the datasets the proposed concatenation method is best suited.

By concatenating successive neighbourhood aggregations and then reducing the dimensionality to a single node representation uniform influence can be achieved.
This is because detail present in the closer neighbourhoods can be combined with the wider awareness of the more receptive neighbourhoods.
Rather than missing larger structure awareness or missing detail JKNs allow for an analysis of both.

For JSGC this leads to two changes to the model and pre-computation.
During pre-computation successive applications of the normalised filter are concatenated together.
A fully-connected layer is then added to the standard SGC to reduce this concatenated representation space to the standard representation space.
During this stage JSGC is able to infer more complex graph structure than SGC.
To combine this with the classifier a single non-linear rectified linear unit layer is introduced.
This non-linearity remains constant regardless of how the model scales and therefore the added potential benefits of the single non-linear layer is deemed negligible.

\section{Repository}

\dirtree{%
.1 \myfolder{black}{}.
.2   \myfolder{red}{README.md}.
.2   \myfolder{red}{requirements.txt}.
.2  \myfolder{blue}{activations.................................extracted activation space}.
.2  \myfolder{blue}{checkpoints..........................................model checkpoints}.
.2  \myfolder{blue}{data...............................................downloaded datasets}.
.2  \myfolder{blue}{output...............................model accuracy and concept output}.
.3  \myfolder{blue}{GCN-BA-Shapes.........................results from this combination}.
.3  \myfolder{blue}{...}.
.2  \myfolder{blue}{run}.
.3 \myfolder{green}{GCN-BA-Shapes.sh..............................experiment run script}.
.3 \myfolder{green}{...}.
.2  \myfolder{blue}{scripts}.
.3 \myfolder{green}{mk\_expr.sh..................................experiment build script}.
.3 \myfolder{green}{...}.
.2  \myfolder{blue}{src}.
.3   \myfolder{red}{eval.py..........................................concept evaluation}.
.3   \myfolder{red}{main.py.........................................ML pipeline control}.
.3   \myfolder{red}{optimise.py.........................................hyperopt script}.
.3   \myfolder{red}{sweep.py......................................hyperparameter sweeps}.
.3  \myfolder{blue}{configs...................................experiment configurations}.
.3  \myfolder{blue}{concepts}.
.4   \myfolder{red}{cluster.py.................................clustering algorithms}.
.4   \myfolder{red}{metrics.py....................completeness and purity algorithms}.
.4   \myfolder{red}{plotting.py..............................visualisation functions}.
.4  \myfolder{blue}{tests}.
.3  \myfolder{blue}{datasets}.
.4   \myfolder{red}{synthetic.py.............................synthetic dataset class}.
.4  \myfolder{blue}{tests}.
.3  \myfolder{blue}{loaders}.
.4   \myfolder{red}{utils.py...............................pre-computation functions}.
.4  \myfolder{blue}{tests}.
.3  \myfolder{blue}{models}.
.4   \myfolder{red}{gcn.py.................................................GCN model}.
.4   \myfolder{red}{sgc.py.................................................SGC model}.
.4   \myfolder{red}{activation\_classifier.py................decision tree classifier}.
.4  \myfolder{blue}{layers.............................................custom layers}.
.3  \myfolder{blue}{wrappers.................................pytorch lightning wrappers}.
}
