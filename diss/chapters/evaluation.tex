\chapter{Evaluation}

\section{Success criteria}

\paragraph{Success Criterion}
The original project proposal (\Aref{ch:proposal}) stated the following three criteria for success:
\begin{enumerate}
    \item 
        Implement SGC and extract the concepts used for each of the synthetic datasets.
        \label{crit1}
    \item 
        Implement GCN and extract the concepts to use as a baseline.
        \label{crit2}
    \item 
        Compare the concepts between SGC and GCN using the metrics of concept completeness and concept purity.
        \label{crit3}
\end{enumerate}

\emph{I completely meet all three success criteria.}
In addition to the above project success criteria, and to aid the analysis of SGC compared to GCN, 
I compare the two models to each other in mean \emph{test accuracy}.

\paragraph{Meeting criterion 1}
\note{reference the implementation.}
\Sref{sec:reproduction} verifies the correctness of the SGC implementation.
\Sref{sec:comp-concept} demonstrates concept extraction on the synthetic datasets.

\paragraph{Meeting criterion 2}
\note{reference the implementation.}
\Sref{sec:reproduction} verifies the correctness of the GCN implementation and demonstrates concept extraction.

\paragraph{Meeting criterion 3}
\note{reference the implementation.}
\Sref{sec:comp-concept} demonstrate the comparison of SGC and GCN using the metrics of concept completeness and purity.
Additionally, \Sref{sec:comp-acc} demonstrates further comparison between the models using the metric of model accuracy.

\section{Methodology}

\subsection{Hyperparameters}
\label{sec:hyperparameters}

\paragraph{Reproduction}
\textit{Magister et al.}\cite{magister2021gcexplainer} use a GCN model to evaluate their proposed graph concept explainer.
The paper trains and evaluates the model on the same 5 synthetic node classification datasets described in \Sref{sec:synth} and therefore the same hyperparameters are used for the GCN baseline.

\textit{Wu et al.}\cite{wu2019simplifying} states that the weight decay parameter for the Planetoid datasets was found using \texttt{hyperopt} over 60 iterations.
This process was repeated for results reproduction however it was found that the hyperparameters for the learning rate were different from those stated.

The hyperparameters for these models are available in tables \ref{tab:GCN-params} and \ref{tab:SGC-reproduction-params}.
Additionally the concept extraction metrics are presented in \ref{tab:GCN-concept-params}.

\begin{table}
    \centering
    \begin{tabular}{c|ccc}
        \textbf{Dataset} &
        \textbf{Layers} &
        \textbf{Learning rate} &
        \textbf{Epochs} \\
        \midrule
        BA Shapes       & 3 $\times$ 20 & 0.001 & 3000 \\
        BA Grid         & 3 $\times$ 20 & 0.001 & 3000 \\
        BA Community    & 6 $\times$ 50 & 0.001 & 6000 \\
        Tree Cycles     & 3 $\times$ 50 & 0.001 & 7000 \\
        Tree Grid       & 7 $\times$ 20 & 0.001 & 10000 \\
        \midrule \\
        REDDIT BINARY   & 4 $\times$ 40 & 0.005 & 3000 \\
        Mutagenicity    & 4 $\times$ 30 & 0.005 & 10000 \\
    \end{tabular}
    \caption{GCN hyperparameters from table 15 in \textit{Magister et al.}\cite{magister2021gcexplainer}. All models use \texttt{ReLU}\note{citation needed?} activation functions between layers with a single linear layer classifier with \texttt{softmax}\note{citation needed?}.}
    \label{tab:GCN-params}
\end{table}


\begin{table}
    \centering
    \begin{tabular}{c|cccc}
        \textbf{Dataset} &
        \textbf{Degree} &
        \textbf{Learning rate} &
        \textbf{Weight decay} &
        \textbf{Epochs} \\
        \midrule
        Cora        & 2 & 0.002 & 0.04 & 100 \\
        Citeseer    & 2 & 0.004 & 0.08 & 100 \\
        PubMed      & 2 & 0.0005 & 0.25 & 100 \\
    \end{tabular}
    \caption{SGC hyperparameters using \texttt{hyperopt}\note{citation needed} as proposed by \textit{Wu et al.}\cite{wu2019simplifying}. All models use \texttt{softmax}\note{citation needed?}.}
    \label{tab:SGC-reproduction-params}
\end{table}


\begin{table}
    \centering
    \begin{tabular}{c|cc}
        \textbf{Dataset} &
        \textbf{Clusters (k)} &
        \textbf{Receptive field (n)} \\
        \midrule
        BA Shapes       & 10 & 2 \\
        BA Grid         & 10 & 3 \\
        BA Community    & 30 & 2 \\
        Tree Cycles     & 10 & 3 \\
        Tree Grid       & 10 & 3 \\
        \midrule
        REDDIT BINARY   & 20 & 1 \\
        Mutagenicity    & 30 & 3 \\
    \end{tabular}
    \caption{Concept extraction parameters used in \textit{Magister et al.}\cite{magister2021gcexplainer}.}
    \label{tab:GCN-concept-params}
\end{table}



\paragraph{New models}
This project proposes multiple new models designed for the different datasets.
The core project defines 5 models based on the SGC architecture as required by criterion \ref{crit1}.
As the underlying graph operator is the same for both GCN and SGC, as described in \Sref{sec:GCN} and \Sref{sec:SGC} the degree of the SGC model can be assumed to be the same as the GCN model for the same dataset.
This the same approach that \textit{Wu et al.}\cite{wu2019simplifying} take to creating there SGC models.

The learning rate for SGC models is likely to be different from the GCN models due to the reformulation.
Furthermore, \textit{Wu et al.}\cite{wu2019simplifying} use weight decay to keep weight values close to $0$ as would be assumed from the multiplication of weight matrices in equation \ref{eq:theta}.
Rather than use stochastic approaches to finding these hyperparameters a sweep of anticipated values is carried out.
This allows for a visualisation of the hyperparameter allowing for further exploration if necessary.
The learning rate is sampled from $\{0.01, 0.001, 0.0001\}$ and the weight decay from $\{1.0, 0.1, 0.01\}$.

The results of these searches is presented in figures \note{include the references the hyperparameter surfaces}.
As can be seen for the majority of hyperparameter searches the specific hyperparameters have minimal to no impact on the resulting model accuracy and therefore $0.01$ is chosen for learning rate and $0.1$ for weight decay.
The hyperparameters chosen are presented in table \ref{tab:SGC-params}

\begin{table}
    \centering
    \begin{tabular}{c|cccc}
        \textbf{Dataset} &
        \textbf{Degree} &
        \textbf{Learning rate} &
        \textbf{Weight decay} &
        \textbf{Epochs} \\
        \midrule
        BA Shapes       & 3 & 0.01 & 0.1 & 3000 \\
        BA Grid         & 3 & 0.01 & 0.1 & 3000 \\
        BA Community    & 6 & 0.001 & 0.1 & 6000 \\
        Tree Cycles     & 3 & 0.01 & 1.0 & 7000 \\
        Tree Grid       & 7 & 0.01 & 0.01 & 10000 \\
    \end{tabular}
    \caption{SGC hyperparameters based on the corresponding GCN hyperparameters in \ref{tab:GCN-params} and using the results of the hyperparameter search in \note{reference when added}. All models use \texttt{softmax}\note{citation needed?}.}
    \label{tab:SGC-params}
\end{table}



\paragraph{Datasets}
The batch sizes for all the datasets match those described in \textit{Ying et al.}\cite{ying2019gnnexplainer} and \textit{Kipf et al.}\cite{kipf2016semi}.
The number of epochs for each dataset matches those proposed in \textit{Magister et al.}\cite{magister2021gcexplainer} and \textit{Wu et al.}\cite{wu2019simplifying} or until convergence for SGC.

\subsection{Model Evaluation}
\label{sec:evaluation}

\paragraph{Concept evaluation}
Criterion \ref{crit3} requires evaluation of the models with respect to concept purity and completeness.
These metrics are discussed in \ref{sec:GCE} and implementation is described in \note{reference when complete}.
On top of the quantitative analysis of the different models concepts lend themselves to qualitative analysis which will mainly focus on visual similarities between the two models.
the quantitative analysis will also help to infer the differences in how the two models infer labels on the input data.

For qualitative analysis only BA Shapes and Mutagenicity will be covered in detail with brief analysis of the other datasets in Appendix \note{reference when complete}.
This extends to extensions where only BA Shapes and Mutagenicity are trained on as proofs of concepts.

It is important to note a number of drawbacks of the GCExplainer in comparing two different models quantitatively.
\begin{enumerate}
    \item 
        Concept purity is calculated only using subgraphs with less than 13 nodes.
        This means that pure quantitative analysis does not represent a full comparison of the two models.
    \item 
        The number of clusters and the receptive field of the concepts can be arbitrarily manipulated to find the highest score.
        To combat this the new SGC models are compared against the same concept extraction parameters presented in table \note{reference when completed}.
    \item 
        \label{nb:accuracy}
        \textit{Ying et al.}\cite{ying2019gnnexplainer} only suggest concept extraction for models that achieve an accuracy of atleast $95\%$ on synthetic datasets. 
\end{enumerate}

The full comparison of concepts requires a qualitative analysis of the extracted concepts.
The concepts produced by SGC can be analysed in isolation to infer how SGC reasons about graphs.
These can then be compared to the concepts produced by GCN to highlight the differences in reasoning and which is easier to understand.
When reproducing results this is done by comparing the analysis suggested by \textit{Magister et al.}\cite{magister2021gcexplainer} to the reproduced concepts.
Furthermore, a visual comparison of the concepts can be made, such as matching published concepts to reproduced concepts.

\paragraph{Accuracy evaluation}
Drawback \ref{nb:accuracy} motivates the additional evaluation metric of accuracy as \Sref{sec:comp-acc} demonstrates that SGC does not meet the desired accuracy.
To evaluate this each synthetic dataset is split into a train and test set using an 80:20 split.
Note that TUDataset\note{citation needed} and Planetoid\note{citation needed} use there own train/test splits.

The synthetic datasets are generated randomly along with the train/test split and thus each random seed produces a new variation of the synthetic dataset.
This means that using the same seeds across different models for the experiments results in the same train/test split.
This also means that although the hyperparameter search evaluates parameters on the test split of the dataset using a different seed to the experiments means that the experiment test splits are effectively unseen.

\subsection{Confidence intervals}
\label{sec:reporting}
The mean accuracy across 10 different initiliasations is reported using $\mu = \sum_i\frac{\text{accuracy}_i}{10}$ as an unbiased estimator of the mean.
The confidence interval of each of the runs uses the unbiased standard deviation estimator $\sigma = \sqrt{\sum_i(\text{accuracy}_i - \mu)/(10 - 1)}$.
For experiments where very high variance is present outliers are removed based on the median, $m$, and the interquartile range, $\text{ITR}$, of the accuracies.\footnote{In the cases where outliers are removed the estimators are adjusted accordingly.}
An outlier is defining as being outside the range $[m - 1.5 \times \text{ITR}, m + 1.5 \times \text{ITR}]$.


\subsection{Reproducibility}
\error{Return to at the end to make sure that all aspects are met.}

\subsection{System specifications}
The experiments are not resource-intensive due to the incredibly small datasets and so carrying out the hyperparameter search and multiple final runs can be completed on my personal machine.
My machine has an AMD Ryzen 7 5700U CPUs @ 1.8GHz with 16 cores wuth 15 Gigabytes of RAM.
The machine does have an AMD ATI Lucienne GPU but due to the fact that \texttt{PyTorch Geometric}\note{citation needed} did not support RoCM I was unable to utilise this.

To speed up the retrieval of experimental results for extensions I utilised a Google Colab Pro account with 1 hyperthreaded Intel Xeon Processor @ 2.3GHz with 1 core and 12 Gigabytes of Ram.
The account also has access to a Tesla K80 GPU with 12GB of RAM.

\note{Include figures demonstrating system use later.}

\section{Results Reproduction}
\label{sec:reproduction}

As discussed in \Sref{sec:testing} the testing strategy includes the reproduction of prior results from \textit{Magister et al.}\cite{magister2021gcexplainer} on GCN and \textit{Wu et al.}\cite{wu2019simplifying} on SGC.
I reproduce all of the experiments from \textit{Magister et al.} and the Planetoid\cite{kipf2016semi} from \textit{Wu et al.} as these are most relevant.

\paragraph{Success criterion}
\Sref{sec:GCN-reproduction} demonstrates an implementation of GCN trained on the synthetic datasets with concept extraction.
Tables \ref{tab:GCN-acc} and \ref{tab:GCN-concepts} and figure \ref{fig:GCN-BA-Shapes} demonstrate the results achieved.
This fulfills the \ref{crit2}$^{nd}$ success criterion.

\paragraph{Method}
In both cases across all datasets the hyperparameters presented in tables \ref{tab:GCN-params} and \ref{tab:SGC-reproduction-params} are used.
Each model, dataset experiment is run 10 times with different randomly pre-selected seeds and the mean and confidence interval are presented in accordance with \Sref{sec:reporting}.
The implementation of SGC is the one presented in \note{reference section when complete} and the implementation of GCN uses the layers provided by \texttt{PyTorch Geometric}\cite{Fey/Lenssen/2019}.

\subsection{SGC}
\begin{table}
    \centering
    \begin{tabular}{c|cc}
        \multicolumn{1}{c}{\textbf{Dataset}} &
        \multicolumn{1}{c}{\textbf{Wu et al.}} &
        \multicolumn{1}{c}{\textbf{My results}} \\
        \midrule
        Cora        & 81.0\% $\pm$ 0.0 & 80.0\% $\pm$ 0.1 \\
        Citeseer    & 71.9\% $\pm$ 0.1 & 71.4\% $\pm$ 0.1 \\
        Pubmed      & 78.9\% $\pm$ 0.0 & 71.3\% $\pm$ 2.1 \\
    \end{tabular}
    \caption{Reproduction of experiments from table x in \textit{Wu et al.}\cite{wu2019simplifying}. The mean accuracy of 10 experiment runs is taken using hyperparameters found by \texttt{hyperopt}\note{citation needed}.}
    \label{tab:GCN-acc}
\end{table}



Table \ref{tab:SGC-reproduction} presents the accuracy achieved by my SGC models compared to the accuracy presented in table 2 of \textit{Wu et al.}\cite{wu2019simplifying}.
As can be seen in both Cora and Citeseer the accuracies are closely correlated, though the accuracy presented by \textit{Wu et al.} are outside of my confidence interval.
Comparitively Pubmed presents a large discrepancy between the published results and the reproduced results with large variation in the reproduced results.
These descrepancies are likely due to the uncertainty in hyperparameters as the exact weight decay constant used is unknown.
Furthermore, the published learning rate of 0.2 yields worse results and so, as demonstrated in table \ref{tab:SGC-reproduction-params}, new learning rates are used.
Based on these considerations I consider my implementation of SGC to be correct.

\subsection{GCN}
\label{sec:GCN-reproduction}
\paragraph{Accuracy}
\begin{table}
    \centering
    \begin{tabular}{c|cc}
        \multicolumn{1}{c}{\textbf{Dataset}} &
        \multicolumn{1}{c}{\textbf{Magister et al.}} &
        \multicolumn{1}{c}{\textbf{My results}} \\
        \midrule
        BA Shapes       & 95.6\% & 98.0\% $\pm$ 2.2 \\
        BA Grid         & 99.0\% & 99.5\% $\pm$ 0.7 \\
        BA Community    & 95.7\% & \% $\pm$  \\
        Tree Cycles     & 96.0\% & 95.8\% $\pm$ 3.2 \\
        Tree Grid       & 95.1\% & 92.5\% $\pm$ 5.4 \\
        \midrule
        REDDIT BINARY   & 89.1\% & 89.1\% $\pm$ 2.2 \\
        Mutagenicity    & 93.0\% & 79.3\% $\pm$ 1.7 \\
    \end{tabular}
    \caption{Comparison of test accuracy published by \textit{Magister et al.}\cite{magister2021gcexplainer} and those achieved by myself.}
    \label{tab:GCN-acc}
\end{table}

%\begin{enumerate}
%    \item[S1] Implementation of SGC
%    \item[S2] Implementation of GCN
%    \item[S3] Implementation of concept evaluation
%\end{enumerate}


Table \ref{tab:GCN-acc} presents the accuracy achieved my GCN models compared to the accuracy published in table 16 of \textit{Magister et al.}\cite{magister2021gcexplainer}.
As can be seen in the majority of cases the reproduced accuracy matches or exceeds the accuracy presented by \textit{Magister et al.}.
Given the small size of the synthetic datasets it is expected that close to 100\% accuracy is achieved which is demonstrated in all but BA Community.
For the real-world datasets the expectation is above 85\% as suggested by \textit{Ying et al.}\cite{ying2019gnnexplainer}.

BA Community is a significant outlier in the results neither reaching the suggested 95\% or reaching a value close to 100\%.
If the model is allowed to train for more than the defined 6000 epochs an accuracy closer to 95\% is achieved.
A likely cause for this discrepancy is the uncertainty in the construction of the community dataset discussed in \note{reference when complete}.
Due to these considerations I consider my implementation of GCN to be correct.

\paragraph{Concept scores}
\begin{table}
    \centering
    \begin{tabular}{c|cccc}
        \multirow{2}{*}{\textbf{Dataset}} &
        \multicolumn{2}{c}{\textbf{Magister et al.}} &
        \multicolumn{2}{c}{\textbf{My results}} \\
        &
        \textbf{Completeness} & 
        \textbf{Purity} & 
        \textbf{Completeness} & 
        \textbf{Purity} \\
        \midrule
        BA Shapes       & 0.964 & 3.375 & 0.929 & 0.000 \\
        BA Grid         & 1.000 & 4.923 & 1.000 & 0.000 \\
        BA Community    & 0.678 & 0.000 & 0.623 & 5.600 \\
        Tree Cycles     & 0.949 & 1.167 & 0.970 & 4.391\\
        Tree Grid       & 0.965 & 3.100 & 0.951 & 1.417\\
        \midrule
        REDDIT BINARY   & 0.713 & 6.968 & 0.746 & 1.429 \\
        Mutagenicity    & 0.967 & 5.400 & 0.626\tablefootnote{See footnote \ref{nb:epochs1}} & 0.375 \\
    \end{tabular}
    \caption{Reproduction of experiments from tables 4 and 5 in \textit{Magister et al.}\cite{magister2021gcexplainer}. The activation space for each experiment selected from the best performing model is used with the mean purity score presented.}
    \label{tab:GCN-concepts}
\end{table}


Table \ref{tab:GCN-concepts} presents the concept scores for each of the top performing GCN models compared to those in tables 4 and 5 in \textit{Magister et al.}.
Only the mean purity is presented across all the models with the minimum and maximum purity available in table \note{reference when complete}. As with the accuracy reproduction the values for concept completeness are closely correlated for the majority of the models.

\fig{erroneous-labels}{A subset of the concepts discovered for Mutagenicity highlighting the erroneous labels. Each row represents a concept and the graphs are coloured according to standard chemical colouring \note{citation needed}.}

Figure \ref{fig:erroneous-labels} demonstrates that the clustering for Mutagenicity focuses on chemical similarity which does not correlate to mutagen similarity.
This means that the concept completeness is likely to be low as demonstrated in \ref{tab:GCN-concepts} though the actual model accuracy may be high.

The average purity is far more erratic with little correlation between the two results which would suggest that the purity score or concept extraction is incorrect.
However, the 13 node cut-off and non-deterministic nature of k-Means\note{citation needed} clustering is likely to be the cause rather than an incorrect implementation.

\fig{GCN-BA-Shapes}{A subset of concepts discovered for BA-Shapes from the best perfomring GCN model. Green nodes highlight the node of interest and pink nodes highlight the neighbourhood used for inference. Each row represents an individual concept.}

\fig{Magister-BA-Shapes}{A subset of the BA Shapes concepts discovered in figures 2, 3 and 5 from \textit{Magister et al.}\cite{magister2021gcexplainer} to demonstrate the full range of labels. Each row represents an individual concept, the same colour system is used as fig. \ref{fig:GCN-BA-Shapes}.}

Figure \ref{fig:GCN-BA-Shapes} presents a subset of the BA Shapes concepts reproduced by the best performing GCN model.
For comparison figure \ref{fig:Magister-BA-Shapes} presents those published in \textit{Magister et al.}
Label 0 is included in both figures to demonstrate that the model does identify the base graph, in this case Barabasi-Albert, but as can be seen this provides little insight into how the model reasons.
The remaining concepts demonstrate the 3 other labels associated with the house motif, as discussed in \Sref{sec:synth}.


As can be seen all the published concepts have an equivalent concept in the reproduced concepts.
The same analysis can be made that the edge attaching the house motif to the base graph is important to the classification of the nodes.
There is also the same distinction in label 1 between a node on the ``inside'' and the ``outside'' where the ``inside'' node has an additional edge attached to the base graph.
This distinction is present in label 2 as well, given that all the nodes present in both the published and reproduced concepts focus on the ``inside'' node.

In both cases, and as expected given the concept purity scores, the concepts related to the motifs are almost completely pure with the exception of label 1 ``inside'' node.
Additionally, all the nodes in the house motif have a unique concept where applicable and this leads to the high completeness score.

Given the visual similarity in concepts between the published and reproduced results I consider the implementation of GCN to be accurate.
Examples of the other synthetic and real datasets is available in Appendix \note{reference when complete}.
\note{There may also be brief comparisons between the figures time permitting.}

\section{Comparison of Accuracy}
\label{sec:comp-acc}
\begin{table}
    \centering
    \begin{tabular}{c|c}
        \textbf{Dataset} & \textbf{Accuracy} \\
        \midrule
        BA Shapes       & 61.4\% $\pm$ 3.3 \\
        BA Grid         & 72.4\% $\pm$ 1.6 \\
        BA Community    & \% $\pm$ \\
        Tree Cycles     & 50.5\% $\pm$ 4.2 \\
        Tree Grid       & 59.2\% $\pm$ 1.6 \\
    \end{tabular}
    \caption{Test accuracy in \% of SGC on each of the synthetic datasets using the hyperparameters in \note{reference when table complete}. As per \textit{Wu et al.}\cite{wu2019simplifying} outliers are removed as defined in \note{reference when complete}.}
    \label{tab:GCN-acc}
\end{table}



Table \ref{tab:SGC-acc} demonstrates the mean accuracies achieved by each of the SGC models using the hyperparameters in \ref{tab:SGC-params}.
As a comparison the accuracies achieved by GCN are presented beside them.
The performanace of SGC is very poor and so random guesses are included as well to demonstrate the limited inference of SGC.

\paragraph{Compared to GCN}
the accuracies achieved by SGC are significantly worse and go against \textit{Wu et al.}'s claim that SGC can match GCN perfomance.
In the cases of the Tree datasets GCN nearly doubles the accuracy and in the case of BA Community GCN is more than $4\times$ as accurate.
Only in the case of BA Grid does SGC achieve a good accuracy in comparison to GCN though even here the difference in accuracy is $27.1$\%.

These poor results suggest that the graph structure inference of SGC is far worse than that of GCN as labels are based only on graph structure.
This is also an explanation for why SGC is able to surpass the accuracy of GCN in the Planetoid\cite{Fey/Lenssen/2019} as these datasets rely heavily on node representations.

\paragraph{Compared to random guesses}
SGC does not perform much better except in the cases of BA Shapes and BA Grid.
In the cases of the Tree datasets this can be attributed to the sparsity of the base graph resulting in less information being aggregated in the pre-computation stage.
Especially in the case of Tree Cycles where the Cycle structure and BST structure share similar sparse connections.

Comparatively the dense BA graph allows for a better distinction between the motifs and the base graph.
This distinction based on degree is attributed to the degree normalisation demonstrated in equation\ref{eq:GCN-as-GNN}.
This advantage is not present in BA Community because of the increase in classes, the need to distinguish the two communities and the possibility of a motif having multiple connections to different base graphs.

These poor results are not due to poor hyperparameter selection as is demonstrated in Appendix \note{reference when complete} and discussed in \Sref{sec:hyperparameters}.
Instead the explanation for the poor performance is due to the lack of proper graph structure inference in SGC.
This is lack of graph structure inference is not in comparison to GCN but rather a fundamental property of SGC.

\section{Comparison of Concepts}
\label{sec:comp-concept}
\begin{table}
    \centering
    \begin{tabular}{c|cccc}
        \multirow{2}{*}{\textbf{Dataset}} &
        \multirow{2}{*}{\textbf{Completeness}} &
        \multicolumn{3}{c}{\textbf{Purity}} \\
        & & \textbf{Max.} & \textbf{Min.} & \textbf{Mean} \\
        \midrule
        BA Shapes       & 0.882 & 0.0 & 5.0 & 1.6 (9)\\
        BA Grid         & 0.843 & 0.0 & 0.0 & 0.0 (2) \\
        BA Community    & 0.264 & 0.0 & 9.0 & 3.5 (6)  \\
        Tree Cycles     & 0.874 & 0.0 & 6.0 & 2.0 (6) \\
        Tree Grid       & 0.890 & 0.0 & 11.0 & 2.1 (8) \\
    \end{tabular}
    \caption{Concept completeness and purity of SGC on each of the synthetic datasets using the hyperparameters in table \ref{tab:SGC-params}. The brackets represent the number of concepts considered for purity, as per \note{reference when complete} these are graphs with less than 13.}
    \label{tab:GCN-concepts}
\end{table}



\section{Extensions}

\subsection{SGC Graph Classification}

\subsection{SGC and GCN Mixed Model}

\subsection{JumpNet style SGC}

