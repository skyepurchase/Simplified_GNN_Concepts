\chapter{Evaluation}

\section{Success criteria}

\paragraph{Success Criterion}
The original project proposal (\Aref{ch:proposal}) stated the following three criteria for success:
\begin{enumerate}
    \item 
        Implement SGC and extract the concepts used for each of the synthetic datasets.
        \label{crit1}
    \item 
        Implement GCN and extract the concepts to use as a baseline.
        \label{crit2}
    \item 
        Compare the concepts between SGC and GCN using the metrics of concept completeness and concept purity.
        \label{crit3}
\end{enumerate}

\emph{I completely meet all three success criteria.}
In addition to the above project success criteria, and to aid the analysis of SGC compared to GCN, 
I compare the two models to each other in mean \emph{test accuracy}.

\paragraph{Meeting criterion 1}
\note{Link to section in which met.}

\paragraph{Meeting criterion 2}
\Sref{sec:reproduction} verifies that the correct implementation of GCN and concept extraction is used. Results from all datasets are achieved with quantitative and qualitative results.

\paragraph{Meeting criterion 3}
\note{Link to section in which met.}

\section{Methodology}

\subsection{Hyperparameters}

\paragraph{Reproduction}
\textit{Magister et al.}\cite{magister2021gcexplainer} use a GCN model to evaluate their proposed graph concept explainer.
The paper trains and evaluates the model on the same 5 synthetic node classification datasets described in \Sref{sec:synth} and therefore the same hyperparameters are used for the GCN baseline.

\textit{Wu et al.}\cite{wu2019simplifying} states that the weight decay parameter for the Planetoid datasets was found using \texttt{hyperopt} over 60 iterations.
This process was repeated for results reproduction however it was found that the hyperparameters for the learning rate were different from those stated.

The hyperparameters for these models are available in tables \ref{tab:GCN-params} and \ref{tab:SGC-reproduction-params}.
Additionally the concept extraction metrics are presented in \ref{tab:GCN-concept-params}.

\begin{table}
    \centering
    \begin{tabular}{c|ccc}
        \textbf{Dataset} &
        \textbf{Layers} &
        \textbf{Learning rate} &
        \textbf{Epochs} \\
        \midrule
        BA Shapes       & 3 $\times$ 20 & 0.001 & 3000 \\
        BA Grid         & 3 $\times$ 20 & 0.001 & 3000 \\
        BA Community    & 6 $\times$ 50 & 0.001 & 6000 \\
        Tree Cycles     & 3 $\times$ 50 & 0.001 & 7000 \\
        Tree Grid       & 7 $\times$ 20 & 0.001 & 10000 \\
        \midrule \\
        REDDIT BINARY   & 4 $\times$ 40 & 0.005 & 3000 \\
        Mutagenicity    & 4 $\times$ 30 & 0.005 & 10000 \\
    \end{tabular}
    \caption{GCN hyperparameters from table 15 in \textit{Magister et al.}\cite{magister2021gcexplainer}. All models use \texttt{ReLU}\note{citation needed?} activation functions between layers with a single linear layer classifier with \texttt{softmax}\note{citation needed?}.}
    \label{tab:GCN-params}
\end{table}


\begin{table}[h]
    \centering
    \captionsetup{width=.9\textwidth}
    \begin{tabular}{c|cccc}
        \textbf{Dataset} &
        \textbf{Degree} &
        \textbf{Learning rate} &
        \textbf{Weight decay} &
        \textbf{Epochs} \\
        \midrule
        Cora        & 2 & 0.002 & 0.04 & 100 \\
        Citeseer    & 2 & 0.004 & 0.08 & 100 \\
        PubMed      & 2 & 0.0005 & 0.25 & 100 \\
    \end{tabular}
    \caption{SGC hyperparameters using \texttt{hyperopt}\cite{bergstra2013making} as proposed by \textit{Wu et al.}\cite{wu2019simplifying}. All models use \texttt{softmax}.}
    \label{tab:SGC-reproduction-params}
\end{table}


\begin{table}[h]
    \centering
    \captionsetup{width=.9\textwidth}
    \begin{tabular}{c|cc}
        \textbf{Dataset} &
        \textbf{Clusters (k)} &
        \textbf{Receptive field (n)} \\
        \midrule
        BA Shapes       & 10 & 2 \\
        BA Grid         & 10 & 3 \\
        BA Community    & 30 & 2 \\
        Tree Cycles     & 10 & 3 \\
        Tree Grid       & 10 & 3 \\
        \midrule
        REDDIT BINARY   & 20 & 1 \\
        Mutagenicity    & 30 & 3 \\
    \end{tabular}
    \caption{Concept extraction parameters used in \textit{Magister et al.}\cite{magister2021gcexplainer}.}
    \label{tab:GCN-concept-params}
\end{table}



\paragraph{New models}
This project proposes multiple new models designed for the different datasets.
The core project defines 5 models based on the SGC architecture as required by criterion \ref{crit1}.
As the underlying graph operator is the same for both GCN and SGC, as described in \Sref{sec:GCN} and \Sref{sec:SGC} the degree of the SGC model can be assumed to be the same as the GCN model for the same dataset.
This the same approach that \textit{Wu et al.}\cite{wu2019simplifying} take to creating there SGC models.

The learning rate for SGC models is likely to be different from the GCN models due to the reformulation.
Furthermore, \textit{Wu et al.}\cite{wu2019simplifying} use weight decay to keep weight values close to $0$ as would be assumed from the multiplication of weight matrices in equation \ref{eq:theta}.
Rather than use stochastic approaches to finding these hyperparameters a sweep of anticipated values is carried out.
This allows for a visualisation of the hyperparameter allowing for further exploration if necessary.
The learning rate is sampled from $\{0.01, 0.001, 0.0001\}$ and the weight decay from $\{1.0, 0.1, 0.01\}$.

The results of these searches is presented in figures \note{include the references the hyperparameter surfaces}.
As can be seen for the majority of hyperparameter searches the specific hyperparameters have minimal to no impact on the resulting model accuracy and therefore $0.01$ is chosen for learning rate and $0.1$ for weight decay.
The hyperparameters chosen are presented in table \ref{tab:SGC-params}

\begin{table}
    \centering
    \begin{tabular}{c|cccc}
        \textbf{Dataset} &
        \textbf{Degree} &
        \textbf{Learning rate} &
        \textbf{Weight decay} &
        \textbf{Epochs} \\
        \midrule
        BA Shapes       & 3 & 0.01 & 0.1 & 3000 \\
        BA Grid         & 3 & 0.01 & 0.1 & 3000 \\
        BA Community    & 6 & 0.001 & 0.1 & 6000 \\
        Tree Cycles     & 3 & 0.01 & 1.0 & 7000 \\
        Tree Grid       & 7 & 0.01 & 0.01 & 10000 \\
    \end{tabular}
    \caption{SGC hyperparameters based on the corresponding GCN hyperparameters in \ref{tab:GCN-params} and using the results of the hyperparameter search in \note{reference when added}. All models use \texttt{softmax}\note{citation needed?}.}
    \label{tab:SGC-params}
\end{table}



\paragraph{Datasets}
The batch sizes for all the datasets match those described in \textit{Ying et al.}\cite{ying2019gnnexplainer} and \textit{Kipf et al.}\cite{kipf2016semi}.
The number of epochs for each dataset matches those proposed in \textit{Magister et al.}\cite{magister2021gcexplainer} and \textit{Wu et al.}\cite{wu2019simplifying} or until convergence for SGC.

\subsection{Model Evaluation}
\label{sec:evaluation}

\paragraph{Concept evaluation}
Criterion \ref{crit3} requires evaluation of the models with respect to concept purity and completeness.
These metrics are discussed in \ref{sec:GCE} and implementation is described in \note{reference when complete}.
On top of the quantitative analysis of the different models concepts lend themselves to qualitative analysis which will mainly focus on visual similarities between the two models.
the quantitative analysis will also help to infer the differences in how the two models infer labels on the input data.

For qualitative analysis only BA Shapes and Mutagenicity will be covered in detail with brief analysis of the other datasets in Appendix \note{reference when complete}.
This extends to extensions where only BA Shapes and Mutagenicity are trained on as proof of concepts.

It is important to note a number of drawbacks of the GCExplainer in comparing two different models quantitatively.
\begin{enumerate}
    \item 
        Concept purity is calculated only using subgraphs with less than 13 nodes.
        This means that pure quantitative analysis does not represent a full comparison of the two models.
    \item 
        The number of clusters and the receptive field of the concepts can be arbitrarily manipulated to find the highest score.
        To combat this the new SGC models are compared against the same concept extraction parameters presented in table \note{reference when completed}.
    \item 
        \textit{Ying et al.}\cite{ying2019gnnexplainer} only suggest concept extraction for models that achieve an accuracy of atleast $95\%$ on synthetic datasets. 
\end{enumerate}

\paragraph{Accuracy evaluation}
The final point motivates the additional evaluation metric of accuracy as \Sref{sec:comparison} demonstrates that SGC does not meet the desired accuracy.
To evaluate this each synthetic dataset is split into a train and test set using an 80:20 split.
Note that TUDataset\note{citation needed} and Planetoid\note{citation needed} use there own train/test splits.

The synthetic datasets are generated randomly along with the train/test split and thus each random seed produces a new variation of the synthetic dataset.
This means that using the same seeds across different models for the experiments results in the same train/test split.
This also means that although the hyperparameter search evaluates parameters on the test split of the dataset using a different seed to the experiments means this the experiment test splits are unseen.

\subsection{Confidence intervals}
\label{sec:reporting}
The mean accuracy across 10 different initiliasations is reported using $\mu = \sum_i\frac{\text{accuracy}_i}{10}$ as an unbiased estimator of the mean.
The confidence interval of each of the runs uses the unbiased standard deviation estimator $\sigma = \sqrt{\sum_i(\text{accuracy}_i - \mu)/(10 - 1)}$.
For runs where very high variance is present outliers are removed based on the median and the interquartile range of the accuracies.\footnote{In the cases where outliers are removed the estimators are adjusted accordingly.}


%\paragraph{Concept evaluation}

\subsection{Reproducibility}
\error{Return to at the end to make sure that all aspects are met.}

\subsection{System specifications}
The experiments are not resource-intensive due to the incredibly small datasets and so carrying out the hyperparameter search and multiple final runs can be completed on my personal machine.
My machine has an AMD Ryzen 7 5700U CPUs @ 1.8GHz with 16 cores wuth 15 Gigabytes of RAM.
The machine does have an AMD ATI Lucienne GPU but due to the fact that \texttt{PyTorch Geometric}\note{citation needed} did not support RoCM I was unable to utilise this.

To speed up the retrieval of experimental results for extensions I utilised a Google Colab Pro account with 1 hyperthreaded Intel Xeon Processor @ 2.3GHz with 1 core and 12 Gigabytes of Ram.
The account also has access to a Tesla K80 GPU with 12GB of RAM.

\note{Include figures demonstrating system use later.}

\section{Results Reproduction}
\label{sec:reproduction}

As discussed in \Sref{sec:testing} the testing strategy includes the reproduction of prior results from \textit{Magister et al.}\cite{magister2021gcexplainer} on GCN and \textit{Wu et al.}\cite{wu2019simplifying} on SGC.
I reproduce all of the experiments from \textit{Magister et al.} and the Planetoid\note{kipf2016semi} from \textit{Wu et al.} as these are most relevant.

\paragraph{Method}
In both cases across all datasets the hyperparameters presented in tables \ref{tab:GCN-params} and \ref{tab:SGC-reproduction-params} are used.
Each model, dataset experiment is run 10 times with different randomly pre-selected seeds and the mean and confidence interval are presented in accordance with \Sref{sec:reporting}.
The implementation of SGC is the one presented in \note{reference section when complete} and the implementation of GCN uses the layers provided by \texttt{PyTorch Geometric}\cite{Fey/Lenssen/2019}.

\paragraph{SGC}
\begin{table}
    \centering
    \begin{tabular}{c|cc}
        \multicolumn{1}{c}{\textbf{Dataset}} &
        \multicolumn{1}{c}{\textbf{Wu et al.}} &
        \multicolumn{1}{c}{\textbf{My results}} \\
        \midrule
        Cora        & 81.0\% $\pm$ 0.0 & 80.0\% $\pm$ 0.1 \\
        Citeseer    & 71.9\% $\pm$ 0.1 & 71.4\% $\pm$ 0.1 \\
        Pubmed      & 78.9\% $\pm$ 0.0 & 71.3\% $\pm$ 2.1 \\
    \end{tabular}
    \caption{Reproduction of experiments from table 2 in \textit{Wu et al.}\cite{wu2019simplifying}. The mean accuracy of 10 experiment runs is taken using hyperparameters found by \texttt{hyperopt}\note{citation needed}.}
    \label{tab:SGC-reproduction}
\end{table}



Table \ref{tab:SGC-reproduction} presents the accuracy achieved by my SGC models compared to the accuracy presented in table 2 of \textit{Wu et al.}\cite{wu2019simplifying}.
As can be seen in both Cora and Citeseer the accuracies are closely correlated, though the accuracy presented by \textit{Wu et al.} are outside of my confidence interval.
Comparitively Pubmed presents a large discrepancy between the published results and the reproduced results with large variation in the reproduced results.
These descrepancies are likely due to the uncertainty in hyperparameters as the exact weight decay constant used is unknown.
Furthermore, the published learning rate of 0.2 yields worse results and so, as demonstrated in table \ref{tab:SGC-reproduction-params}, new learning rates are used.
Based on these considerations I consider my implementation of SGC to be correct.

\paragraph{GCN}
\begin{table}
    \centering
    \begin{tabular}{c|cc}
        \multicolumn{1}{c}{\textbf{Dataset}} &
        \multicolumn{1}{c}{\textbf{Magister et al.}} &
        \multicolumn{1}{c}{\textbf{My results}} \\
        \midrule
        BA Shapes       & 95.6\% & 98.0\% $\pm$ 2.2 \\
        BA Grid         & 99.0\% & 99.5\% $\pm$ 0.7 \\
        BA Community    & 95.7\% & 86.3\%\tablefootnote{\label{nb:epochs1}Using the same epochs stated in \textit{Magister et al.}\cite{magister2021gcexplainer}. More epochs yields a closer value.} $\pm$ 3.3 \\
        Tree Cycles     & 96.0\% & 95.8\% $\pm$ 3.2 \\
        Tree Grid       & 95.1\% & 92.5\% $\pm$ 5.4 \\
        \midrule
        REDDIT BINARY   & 89.1\% & 89.1\% $\pm$ 2.2 \\
        Mutagenicity    & 93.0\% & 93.8\% $\pm$ 1.6 \\
    \end{tabular}
    \caption{Reproduction of experiments from table 16 in \textit{Magister et al.}\cite{magister2021gcexplainer}. The mean test accuracy of 10 experiment runs is taken using \textit{Magister et al.}'s hyperparameters.}
    \label{tab:GCN-acc}
\end{table}


\begin{table}
    \centering
    \begin{tabular}{c|cccc}
        \multirow{2}{*}{\textbf{Dataset}} &
        \multicolumn{2}{c}{\textbf{Magister et al.}} &
        \multicolumn{2}{c}{\textbf{My results}} \\
        &
        \textbf{Completeness} & 
        \textbf{Purity} & 
        \textbf{Completeness} & 
        \textbf{Purity} \\
        \midrule
        BA Shapes       & 0.964 & 3.375 & 0.929 & 0.000 \\
        BA Grid         & 1.000 & 4.923 & 1.000 & 0.000 \\
        BA Community    & 0.678 & 0.000 & 0.623 & 5.600 \\
        Tree Cycles     & 0.949 & 1.167 & 0.970 & 4.391\\
        Tree Grid       & 0.965 & 3.100 & 0.951 & 1.417\\
        \midrule
        REDDIT BINARY   & 0.713 & 6.968 & 0.746 & 1.429 \\
        Mutagenicity    & 0.967 & 5.400 & 0.626 & 0.375 \\
    \end{tabular}
    \caption{Reproduction of experiments from tables 4 and 5 in \textit{Magister et al.}\cite{magister2021gcexplainer}. The activation space for each experiment selected from the best performing model is used with the mean purity score presented.}
    \label{tab:GCN-concepts}
\end{table}



Table \ref{tab:GCN-acc} presents the accuracy achieved my GCN models compared to the accuracy published in table 16 of \textit{Magister et al.}\cite{magister2021gcexplainer}.
As can be seen in the majority of cases the reproduced accuracy matches or exceeds the accuracy presented by \textit{Magister et al.}.

\section{Comparison of Accuracy}
\label{sec:comparison}

\begin{table}
    \centering
    \begin{tabular}{c|c}
        \textbf{Dataset} & \textbf{Accuracy} \\
        \midrule
        BA Shapes       & 61.4\% $\pm$ 3.3 \\
        BA Grid         & 72.4\% $\pm$ 1.6 \\
        BA Community    & 20.6\% $\pm$ 1.5 \\
        Tree Cycles     & 50.5\% $\pm$ 4.2 \\
        Tree Grid       & 59.2\% $\pm$ 1.6 \\
    \end{tabular}
    \caption{Test accuracy in \% of SGC on each of the synthetic datasets using the hyperparameters in table \ref{tab:SGC-params}. As per \textit{Wu et al.}\cite{wu2019simplifying} outliers are removed as defined in \note{reference when complete}.}
    \label{tab:GCN-acc}
\end{table}



\section{Comparison of Concept Scores}

\begin{table}
    \centering
    \begin{tabular}{c|cccc}
        \multirow{2}{*}{\textbf{Dataset}} &
        \multirow{2}{*}{\textbf{Completeness}} &
        \multicolumn{3}{c}{\textbf{Purity}} \\
        & & \textbf{Max.} & \textbf{Min.} & \textbf{Mean} \\
        \midrule
        BA Shapes       & 0.882 & 0.0 & 5.0 & 1.6 (9)\\
        BA Grid         & 0.843 & 0.0 & 0.0 & 0.0 (2) \\
        BA Community    & 0.264 & 0.0 & 9.0 & 3.5 (6)  \\
        Tree Cycles     & 0.874 & 0.0 & 6.0 & 2.0 (6) \\
        Tree Grid       & 0.890 & 0.0 & 11.0 & 2.1 (8) \\
    \end{tabular}
    \caption{Concept completeness and purity of SGC on each of the synthetic datasets using the hyperparameters in table \ref{tab:SGC-params}. The brackets represent the number of concepts considered for purity, as per \note{reference when complete} these are graphs with less than 13.}
    \label{tab:GCN-concepts}
\end{table}



\section{Extensions}

\subsection{SGC Graph Classification}

\subsection{SGC and GCN Mixed Model}

\subsection{JumpNet style SGC}

