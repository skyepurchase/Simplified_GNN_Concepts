\chapter{Evaluation}

\section{Success criteria}

\paragraph{Success Criterion}
The original project proposal (\Aref{ch:proposal}) stated the following three criteria for success:
\begin{enumerate}
    \item Implement SGC and extract the concepts used for each of the datasets.
    \item Implement GCN and extract the concepts to use as a baseline.
    \item Compare the concepts between SGC and GCN using the metrics of concept completeness and concept purity.
\end{enumerate}

\emph{I completely meet all three success criteria.}
In addition to the above project success criteria, and to aid the analysis of SGC compared to GCN, 
I compare the two models to each other in \emph{accuracy}.

\paragraph{Meeting criterion 1}
\note{Link to section in which met.}

\paragraph{Meeting criterion 2}
\note{Link to section in which met.}

\paragraph{Meeting criterion 3}
\note{Link to section in which met.}

\section{Methodology}

\subsection{Hyperparameters}

\paragraph{Reproduction}
\textit{Magister et al.}\cite{magister2021gcexplainer} use a GCN model to evaluate their proposed graph concept explainer.
The paper trains and evaluates the model on the same 5 synthetic node classification datasets described in \Sref{sec:synth} and therefore the same hyperparameters are used for the GCN baseline.

\textit{Wu et al.}\cite{wu2019simplifying} states that the weight decay parameter for the Planetoid datasets was found using \texttt{hyperopt} over 60 iterations.
This process was repeated for results reproduction however it was found that the hyperparameters for the learning rate were different from those stated.

The hyperparameters for these models is available in Table \error{create the table}.

\paragraph{New models}
This project proposes multiple new models designed for the different datasets.

\subsection{Model Evaluation}

\subsection{Concept Evaluation}

\subsection{Reproducibility}

\subsection{Specifications}

\section{Results Reproduction}

\section{Comparison of Accuracy}

\section{Comparison of Concept Scores}

\section{Extensions}

\subsection{SGC Graph Classification}

\subsection{SGC and GCN Mixed Model}

\subsection{JumpNet style SGC}

\subsection{Tunable Dataset}
