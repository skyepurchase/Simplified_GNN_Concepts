\chapter{Abbreviations and terms}

\paragraph{AMI}
adjusted mutual information.

\paragraph{BA}
barabasi-albert.

\paragraph{BST}
binary symmetric tree.

\paragraph{GCExplainer}
graph concept explainer --- GNN explainability tool developed by \textit{Magister et al.}~\cite{magister2021gcexplainer}.

\paragraph{GCN}
graph convolution network --- GNN proposed by \textit{Kipf et al.}~\cite{kipf2016semi} that uses the idea of convolutions on graphs.

\paragraph{GED}
graph edit distance.

\paragraph{GNN}
graph neural network --- a neural network archictecture used for processing graph data.

\paragraph{GRL}
graph representation learning.

\paragraph{JSGC}
jump SGC --- a JKN that combines SGC precomputations of different degrees.

\paragraph{JKN}
jumping knowledge network --- a GNN archictecture proposed by \textit{Xu et al.}~\cite{xu2018representation} that can selectively combine different layer aggregations.

\paragraph{MLP}
multi-layer perceptron.

\paragraph{NN}
neural network.

\paragraph{SGC}
simple graph convolution --- a linearised version of GCN proposed by \textit{Wu et al.}~\cite{wu2019simplifying}.

\paragraph{SGCN}
SGC and GCN --- A mixed model of SGC and GCN starting with SGC precomputations before a GCN.

\paragraph{t-SNE}
t-distributed stochastic neighbour embedding.

\paragraph{linear GNN}
A GNN archictecture which includes no non-linear layers and is instead a series of different graph or matrix operations.

\paragraph{quasi-linear GNN}
A GNN archictecture which includes one non-linear layer separating linear graph representation learning from classification.

