\chapter{Datasets}
\label{app:datasets}

\fig{inductive-vs-transductive}{(a) demonstrates transductive learning where bold outlines means that the node features are accessible to the model during training. The solid lines represent the node labels used during training and the dashed lines represent nodes used during testing. (b) demonstrates inductive learning the same system is used as with (a) however now the nodes during testing are not bold and thus not seen during training. As can be seen in (a) removing the dashed nodes would result in all training nodes being disconnected which is not the case in (b).}

\section{Transductive datasets}
These datasets are such that during training the node representations of nodes outside of the training set are ``seen'' by the model but importantly the model does not ``see'' the labels for these nodes.
The model is tested on the labels of these ``unseen'' nodes without access to the training sets labels.
This is because the training, validation and test sets are sampled from the nodes of a single graph, if only these nodes where given to the model the structure of the graph is lost and a large proportion of the nodes will no longer be connected.
This dataset forces the model to transfer knowledge about existing (representation, label) pairs to unlabelled nodes.
This concept is present in Figure \ref{fig:inductive-vs-transductive}(a).

\section{Inductive datasets}
These datasets are such that the nodes ``seen'' during training, validation and testing are completely disjoint.
This is generally the case in graph classification tasks where each data point is a unique graph with a classification thus the test set is a new set of graphs with completely unseen nodes.
This dataset forces the model to induce information of relationships between nodes and optimal representations.
This concept is present in Figure \ref{fig:inductive-vs-transductive}(b).

